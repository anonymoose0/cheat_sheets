% Copyright (c)  2025  anonymoose0.
%    Permission is granted to copy, distribute and/or modify this document
%    under the terms of the GNU Free Documentation License, Version 1.3
%    or any later version published by the Free Software Foundation;
%    with no Invariant Sections, no Front-Cover Texts, and no Back-Cover Texts.
%    A copy of the license is included in the section entitled "GNU
%    Free Documentation License".

\documentclass[12pt]{article}
\usepackage[utf8]{inputenc}
\usepackage{multirow}
\usepackage{stackengine}
\usepackage[left = 0.79 in, right = 0.79 in, top = 0.79 in, bottom = 0.79 in]{geometry}

\begin{document}
\begin{center}

\section*{\underline{Articles and Endings}}

\begin{tabular}{ | c | c | c | c | c | }
    \hline
     & \underline{\textbf{Maskulin}} & \underline{\textbf{Feminin}} & \underline{\textbf{Neuter}} & \underline{\textbf{Mehrzahl}} \\
    \hline
    \underline{\textbf{Nominativ}} & Der \textbar { -} : -e & Die \textbar { -e} : -e & Das \textbar { -} : -e & Die \textbar { -e} : -en \\  
    \hline
    \underline{\textbf{Akkusitiv}} & Den \textbar { -en} : -en & Die \textbar { -e} : -e & Das \textbar { -} : -e & Die \textbar { -e} : -en \\
    \hline
    \underline{\textbf{Dativ}} & Dem \textbar { -em} : -en & Der \textbar { -er} : -en & Dem \textbar { -em} : -en & Den \textbar { -en} : -en \\
    \hline
    \underline{\textbf{Genitiv}} & Des \textbar { -es} : -en & Der \textbar { -er} : en & Des \textbar { -es} : -en & Der \textbar { -er} : en \\
    \hline
\end{tabular}

\smallskip

\begin{tabular}{cc}
\begin{tabular}{ | c | c | }
    \hline
    \multicolumn{2}{ | c | }{\underline{\textbf{Possessives}}} \\
    \hline
    \underline{\textbf{English}} & \underline{\textbf{Deutsch}} \\
    \hline
    my & mein- \\
    \hline
    your & dein- \\
    \hline
    his & sein- \\
    \hline
    her & ihr- \\
    \hline
    its & sein- \\
    \hline
    our & unser- \\
    \hline 
    y'alls & euer- \\
    \hline
    their & ihr- \\
    \hline
    Your (formal) & Ihr- \\
    \hline
\end{tabular}

&

\begin{tabular}{ | c | c |}
    \hline
    \multicolumn{2}{ | c | }{\underline{\textbf{Der Words}}} \\
    \hline
    \underline{\textbf{English}} & \underline{\textbf{Deutsch}} \\
    \hline
    this & dies- \\
    \hline
    that & jen- \\
    \hline
    every/each & jed- \\
    \hline
    many & manch- \\
    \hline
    such & solch- \\
    \hline
    which & welch- \\
    \hline
    all & all- \\
    \hline
\end{tabular}
\end{tabular}



\section*{\underline{Pronouns}}

\begin{tabular}{ | c | c | c | c |}
    \hline
    \underline{\textbf{English (subjective/objective)}} & \underline{\textbf{Nominativ}} & \underline{\textbf{Akkusitiv}} & \underline{\textbf{Dativ}} \\
    \hline
    I/me & ich & mich & mit \\
    \hline
    you & du & dich & dir \\
    \hline
    he/him & er & ihn & ihm \\
    \hline
    she/her & \multicolumn{2}{ | c | }{sie} & ihr \\
    \hline
    it & \multicolumn{2}{ | c | }{es} & ihm \\
    \hline
    we/us & wir & \multicolumn{2}{ | c | }{uns} \\
    \hline
    y'all & ihr & \multicolumn{2}{ | c | }{euch} \\
    \hline
    they/them & \multicolumn{2}{ | c | }{sie} & ihnen \\
    \hline
    You (formal) & \multicolumn{2}{ | c | }{Sie} & Ihnen \\
    \hline
\end{tabular}

\smallskip

\begin{tabular}{ | c | c | c |}
    \hline
    \multicolumn{3}{ | c | }{\underline{\textbf{Reflexives}}} \\
    \hline
    \underline{\textbf{English}} & \underline{\textbf{Akkusitiv}} & \underline{\textbf{Dativ}} \\
    \hline
    myself & mich & mir \\
    \hline
    yourself & dich & dir \\
    \hline
    himself & \multicolumn{2}{ | c | }{\multirow{3}{*}{\Huge{sich}}}  \\
    \cline{1-1}
    herself & \multicolumn{2}{ c |}{} \\
    \cline{1-1}
    itself & \multicolumn{2}{ c |}{} \\
    \hline
    ourselves & \multicolumn{2}{ | c | }{uns} \\
    \hline
    y'allselves & \multicolumn{2}{ | c | }{euch} \\
    \hline
    themselves & \multicolumn{2}{ | c | }{sich} \\
    \hline
    Youself (formal) & \multicolumn{2}{ | c | }{Sich} \\
    \hline
\end{tabular}



\section*{\underline{Prepositions}}

\begin{tabular}{ | c | c | c | c | c | c | c | c | }
    \hline
    \multicolumn{2}{ | c |}{\underline{\textbf{Akkusitiv}}} & \multicolumn{2}{ | c |}{\underline{\textbf{Beide}}} & \multicolumn{2}{ | c |}{\underline{\textbf{Dativ}}} & \multicolumn{2}{ | c |}{\underline{\textbf{Genitiv}}} \\
    \hline
    \underline{\textbf{English}} & \underline{\textbf{Deutsch}} & \underline{\textbf{English}} & \underline{\textbf{Deutsch}} & \underline{\textbf{English}} & \underline{\textbf{Deutsch}} & \underline{\textbf{English}} & \underline{\textbf{Deutsch}} \\
    \hline
    through & durch & \stackanchor{on}{(horizontal)} & auf & out of & aus & instead of & anstatt \\
    \hline
    for & für & \stackanchor{on}{\stackanchor{(vertical/}{aquious)}} & an & except for & außer & despite & trotz \\
    \hline
    against & gegen & over & über & \stackanchor{by/at/}{near} & bei & while & während \\
    \hline
    without & ohne & under & unter & with & mit & because of & wegen \\
    \hline
    at/around & um & behind & hinter & after/to & nach & outside of & außerhalb \\
    \hline
    until & bis & between & zwischen & since & seit & inside of & innerhalb \\
    \hline
    along & entlang & in & in & of/from & von & & \\
    \hline
     & & in front of & vor & to & zu & & \\
    \hline
     & & beside & neben & about & gegen & & \\
     \hline
\end{tabular}



\section*{\underline{Question Words}}

\begin{tabular}{ | c | c | c | c | c | c | c | c | }
    \hline
    \multicolumn{2}{ | c | }{\underline{\textbf{Nominativ}}} & \multicolumn{2}{ | c | }{\underline{\textbf{Akkusitiv}}} & \multicolumn{2}{ | c | }{\underline{\textbf{Dativ}}} & \multicolumn{2}{ | c | }{\underline{\textbf{Genitiv}}} \\
    \hline
    \underline{\textbf{English}} & \underline{\textbf{Deutsch}} & \underline{\textbf{English}} & \underline{\textbf{Deutsch}} & \underline{\textbf{English}} & \underline{\textbf{Deutsch}} & \underline{\textbf{English}} & \underline{\textbf{Deutsch}} \\
    \hline
    who & wer & whom & wen & whom & wem & whose & wessen \\
    \hline
    what & was & & & & & \stackanchor{for}{what reason} & weswegen \\
    \hline
    where & wo & & & & & & \\
    \hline
    why & warum & & & & & & \\
    \hline
    how & wie & & & & & & \\
    \hline
    which & welch- & & & & & & \\
    \hline
\end{tabular}



\section*{\underline{Verbs}}

\begin{tabular}{cc}
\begin{tabular}{ | c | c |}
    \hline
    \multicolumn{2}{ | c | }{\underline{\textbf{Modal Verbs}}} \\
    \hline
    \underline{\textbf{English}} & \underline{\textbf{Deutsch}} \\
    \hline
    to like & mögen \\
    \hline
    to have to, must & müssen \\
    \hline
    to be allowed to, may & dürfen \\
    \hline
    to be able to, can & können \\
    \hline
    to be supposed to, should & sollen \\
    \hline
    to want & wollen \\
    \hline
\end{tabular}

&

\begin{tabular}{ | c | c |}
    \hline
    \multicolumn{2}{ | c | }{\underline{\textbf{Auxillary Verbs}}} \\
    \hline
    \underline{\textbf{English}} & \underline{\textbf{Deutsch}} \\    
    \hline
    to be & sein \\
    \hline
    to have & haben \\
    \hline
    to become & werden \\
    \hline
\end{tabular}
\end{tabular}


\section*{\underline{Conjunctions}}

\begin{tabular}{ | c | c | c | c | }
    \hline
    \multicolumn{2}{ | c | }{\underline{\textbf{Coordinating}}} & \multicolumn{2}{ | c | }{\underline{\textbf{Subordinating}}} \\
    \hline
    \underline{\textbf{English}} & \underline{\textbf{Deutsch}} & \underline{\textbf{English}} & \underline{\textbf{Deutsch}} \\
    \hline
    and & und & when/if & wenn \\
    \hline
    or & oder & in case & falls \\
    \hline
    because & denn & if (boolean) & ob \\
    \hline
    but & aber & as if & als ob \\
    \hline
    \stackanchor{but/however/}{rather} & sondern & although & \stackanchor{obwohl/}{obschon/obgleich} \\
    \hline
    alas & allein & because & weil/da \\
    \hline
    however & doch & so that & damit \\
    \hline
    with relation to & beziehungsweise & that & dass \\
    \hline
     & & by means of & dadurch dass \\
    \hline
     & & \stackanchor{while/}{meanwhile/whereas} & wohingegen \\
    \hline
     & & before & bevor/ehe \\
    \hline
     & & until & bis \\
    \hline
     & & while & während \\
    \hline
     & & as long as & solange \\
    \hline
     & & as soon as & sobald \\
    \hline
     & & as often as & sooft \\
    \hline
     & & after that & nachdem \\
    \hline
     & & since then & seit/seitdem \\
     \hline
\end{tabular}



\end{center}
\end{document}