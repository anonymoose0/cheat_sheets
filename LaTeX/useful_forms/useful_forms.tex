\documentclass[12pt]{article}
\usepackage[utf8]{inputenc}
\usepackage{amsmath, amssymb, amsthm}
\usepackage{fullpage}

\begin{document}
\twocolumn

\section*{\underline{Biology}}
\textbf{Hardy-Weinberg Equilibrium:}
    \\ \( p+q=1 \)
    \\ \( p^{2}+2pq+q^{2}=1 \)



\section*{\underline{Chemistry}}
\textbf{Specific Heat:}
    \\ \( q = sm \Delta t \)
\\ \textbf{Internal Energy:}
    \\ \( \Delta E = q + w \)
\\ \textbf{Definition of Density:}
    \\ \( d = \displaystyle\frac{ m }{ V } \)
\\ \textbf{Definition of Pressure:}
    \\ \( P = \displaystyle\frac{ \vec{ F } }{ A } \)
\\ \textbf{Boyle’s Law:}
    \\ \( V_1 P_1 = V_2 P_2 \)
\\ \textbf{Charles's Law:}
    \\ \( \displaystyle\frac{ V_1 }{ T_1 } = \displaystyle\frac{ V_2 }{ V_2 } \)
\\ \textbf{Gay-Lussac’s Law:}
    \\ \( \displaystyle\frac{ P_1 }{ T_1 } = \displaystyle\frac{ P_2 }{ T_2 } \)
\\ \textbf{Combined Gas Law:}
    \\ \( \displaystyle\frac{ V_1 P_1 }{ T_1 } = \displaystyle\frac{ V_2 P_2 }{ T_2 } \)
\\ \textbf{Ideal Gas Law:}
    \\ \( PV = nRT = \displaystyle\frac{ mRT }{ MM } \)
\\ \textbf{Dalton's Law:}
    \\ \( P_{total} = \displaystyle\sum{ P_i } \)
\\ \textbf{Definition of Molarity:}
    \\ \( M = \displaystyle\frac{ n_{solute} }{ V_{solution} } \)
\\ \textbf{Definition of Molality:}
    \\ \( \bar{ m } = \displaystyle\frac{ n_{solute} }{ m_{solvent} } \)
\\ \textbf{Dilution:}
    \\ \( M_1 V_1 = M_2 V_2 \)
\\ \textbf{Neutralization:}
    \\ \( N_A V_A = N_B V_B \)
\\ \textbf{Definition of pH and pOH:}
    \\ \( pH = -log_{10}{ H^+ } \)
    \\ \( pOH = -log_{10}{ OH^- } \)
\\ \textbf{Enthalpy of Formation:}
    \\ \( \Delta H^{\circ} = \displaystyle\sum{ n H^{\circ}_f } - \displaystyle\sum{ m H^{\circ}_f } \)
\\ \textbf{Effective Nuclear Charge:}
    \\ \( Z_{eff} = Z - S \)
\\ \textbf{Graham’s Law of Effusion:}
    \\ \( \displaystyle\frac{ r_1 }{ r_2 } = \displaystyle\frac{ \sqrt{ M_2 } }{ \sqrt{ M_1 } } \)
\\ \textbf{Rydberg Formula:}
    \\ \( E = \displaystyle\frac{ hc }{ \lambda } \)
\\ \textbf{Plank’s Equation:}
    \\ \( E = h \nu \)
\\ \textbf{Wavelength-Frequency Equation:}
    \\ \( c = \nu \lambda \)
\\ \textbf{Raoult’s Law:}
    \\ \( P_{solution} = \chi_{solvent} \cdot P_{solvent} \)
\\ \textbf{Boiling Point Elevation:}
    \\ \( \Delta T_{bp} = k_b \bar{m} i \)
\\ \textbf{Freezing Point Depression:}
    \\ \( \Delta T_{fp} = k_f \bar{m} i \)
\\ \textbf{Osmotic Pressure:}
    \\ \( \pi = iMRT \)
\\ \textbf{Average Reaction Rate:}
    \\ \( \text{Rate} = \displaystyle\frac{ - \Delta [ R ] }{ \Delta t } = \displaystyle\frac{ \Delta [ P ] }{ \Delta t } \)
\\ \textbf{Reaction Rate Law:}
    \\ \( \text{Rate} = k [ A ]^m [ B ]^n \)
\\ \textbf{Integrated Rate Law of a Zero or First \\Order Reaction:}
    \\ \( \ln[ A ]_t = -kt + \ln[ A ]_0 \)
\\ \textbf{Integrated Rate Law of a Second or Higher Order Reaction:}
    \\ \( \displaystyle\frac{ 1 }{ [ A ]_t } = kt + \displaystyle\frac{ 1 }{ [ A ]_0 } \)
\\ \textbf{Activation Energy of a Reverse \\Reaction:}
    \\ \( E_{a \text{ (Reverse)}} = \Delta E_{\text{(Reverse)}} + E_{a \text{ (Forward)}} \)
\\ \textbf{Arrhenius Equation:}
    \\ \( k = A e^{\frac{ -E_a }{ RT }} \)
\\ \textbf{Half Life of a First Order Reaction:}
    \\ \( t_{\frac{ 1 }{ 2 }} = \displaystyle\frac{ \ln 2 }{ k } \)
\\ \textbf{Half Life of a Second or Higher Order Reaction:}
    \\ \( t_{\frac{ 1 }{ 2 }} = \displaystyle\frac{ 1 }{ k [ A ]_0 } \)
\\ \textbf{Reaction Catalysis:}
    \\ \( \ln \left( \displaystyle\frac{ k_1 }{ k_2 } \right) = \displaystyle\frac{ E_a }{ R } \left[ \displaystyle\frac{ 1 }{ T_2 } - \displaystyle\frac{ 1 }{ T_1 } \right] \)
\\ \textbf{Equilibrium Constant in Terms of \\Concentration:}
    \\ \( K_c = \displaystyle\frac{ [ \text{Products} ]^{\text{Coefficient}} }{ [ \text{Reactants} ]^{\text{Coefficient}} } \)
\\ \textbf{Equilibrium Constant in Terms of Pressure:}
    \\ \( K_P = \displaystyle\frac{ ( P_{\text{products}} )^{\text{Coefficient}} }{ ( P_{\text{reactants}} )^{\text{Coefficient}} } \)
\\ \textbf{The Relationship between the two \\Equilibrium Constants:}
    \\ \( K_P = K_c ( RT )^{\Delta n} \)



\section*{\underline{Physics}}\
\textbf{Mass-Energy Equivalence:}
    \\ \( E = mc^2 \)
    
\subsection*{Newtonian Mechanics}
\textbf{Newton’s Second Law:}
    \\ \( \vec{ F }_{\text{net}} = m_{\text{sys}} \vec{ a } = \displaystyle\frac{ d \vec{ p } }{ dt } \)
\\ \textbf{Newton’s Third Law:  }
    \\ \( \vec{ F }_{a \rightarrow{b}} = - \vec{ F }_{b \rightarrow{a}} \)
\\ \textbf{Definitions of Displacement, Velocity, and Acceleration:}
    \\ \( \Delta x = x_f - x_i \)
    \\ \( \vec{ v } = \displaystyle\frac{ \Delta x }{ \Delta t } \)
    \\ \( \vec{ a } = \displaystyle\frac{ \Delta \vec{ v } }{ \Delta t } \)
\\ \textbf{Displacement with Constant Acceleration:}
    \\ \( x_f = x_i + \vec{ v }_x \Delta t + \displaystyle\frac{ \vec{ a }_x ( \Delta t )^2 }{ 2 } \)
\\ \textbf{Velocity with Constant Acceleration:}
    \\ \( \vec{ v }_{xf} = \vec{ v }_{xi} + \vec{ a }_x \Delta t \)
\\ \textbf{Velocity-Displacement Relation with Constant Acceleration:}
    \\ \( \vec{ v }_{xf}^2 = \vec{ v }_{xi}^2 + 2 \vec{ a }_x \Delta x \)
\\ \textbf{Vector Equations:}
    \\ \( \vec{ A }_x = \vec{ A } \cos \theta \)
    \\ \( \vec{ A }_y = \vec{ A } \sin \theta \)
    \\ \( \vec{ A } = \sqrt{ \vec{ A }_{x}^2 + \vec{ A }_{y}^2 } \)
    \\ \( \theta = \arctan \displaystyle\frac{ \vec{ A }_y }{ \vec{ A }_x } \)
\\ \textbf{Center of Mass:}
    \\ \( x_{cm} = \displaystyle\frac{ \displaystyle\sum{ m_i x_i } }{ \displaystyle\sum{ m_i } } \)
\\ \textbf{Definition of Weight:}
    \\ \( \vec{ F }_w = m( g + \vec{ a }_y ) = -\vec{ F }_n \)
\\ \textbf{Maximum Static Friction:}
    \\ \( \vec{ F }_{\text{sf max}} = \mu \vec{ F }_n \)
\\ \textbf{Kinetic Friction:}
    \\ \( \vec{ F }_{\text{kf}} = \mu_k \vec{ F }_n \)
\\ \textbf{Hooke's Law:}
    \\ \( \vec{ F }_{\text{sp x}} = -k \Delta x \)
\\ \textbf{Newton's Law of Gravitation:}
    \\ \( \vec{ F }_g = \displaystyle\frac{ G m_1 m_2 }{ r^2 } \)
\\ \textbf{Kepler's Third Law:}
    \\ \( t^2 = \displaystyle\frac{ 4 \pi^2 R^3 }{ MG } \)
\\ \textbf{Time to Orbit:}
    \\ \( t = \displaystyle\frac{ 2 \pi r }{ \vec{ v } } \)
\\ \textbf{Minimum Velocity to Orbit:}
    \\ \( \vec{ v }_{\text{min}} = \sqrt{ gr } \)
\\ \textbf{Circular Acceleration:}
    \\ \( \vec{ a }_c = \displaystyle\frac{ \vec{ v }^2 }{ r } \)
\\ \textbf{Work:}
    \\ \( w = \vec{ F }d \cos \theta \)
\\ \textbf{Translational Kinetic Energy:}
    \\ \( k = \displaystyle\frac{ m\vec{ v }^2 }{ 2 } \)
\\ \textbf{Gravitational Potential Energy:}
    \\ \( U_g = mgy \)
\\ \textbf{Elastic Potential Energy:}
    \\ \( U_s = \displaystyle\frac{ k \Delta x^2 }{ 2 } \)
\\ \textbf{Work-Energy Theorem:}
    \\ \( w = \Delta k \)
\\ \textbf{Definition of Power:}
    \\ \( P = \displaystyle\frac{ \Delta E }{ \Delta t } = \displaystyle\frac{ w }{ \Delta t } = \vec{ F } \vec{ v } \cos \theta \)
\\ \textbf{Definition of Impulse:}
    \\ \( \vec{ J } = \vec{ F }_{\text{avg}} \Delta t \)
\\ \textbf{Definition of Momentum:}
    \\ \( \vec{ p } = m\vec{ v } \)
\\ \textbf{Conservation of Momentum:}
    \\ \( \vec{ p }_f - \vec{ p }_i = 0 \)
\\ \textbf{Impulse-Momentum Theorem:}
    \\ \( \vec{ J } = \Delta \vec{ p } = m \Delta \vec{ v } = \vec{ F } \Delta t \)
\\ \textbf{Orbital Velocity:}
    \\ \( \vec{ v } = \sqrt{ \displaystyle\frac{ Gm }{ r } } \)
\\ \textbf{Orbital Gravitational Potential Energy:}
    \\ \( U_g = \displaystyle\frac{ -G m_1 m_2 }{ r } \)
\\ \textbf{Escape Velocity:}
    \\ \( \vec{ v }_{\text{esc}} = \sqrt{ \displaystyle\frac{ 2GM }{ r } } \)
\\ \textbf{Period of a Pendulum:}
    \\ \( t_p = 2 \pi \sqrt{ \displaystyle\frac{ l }{ g } } \)
\\ \textbf{Period of a Spring:}
    \\ \( t_s = 2 \pi \sqrt{ \displaystyle\frac{ m }{ k } } \)

\subsection*{Rotational Mechanics}
\textbf{Definitions of Angular Displacement, Velocity, and Acceleration:}
    \\ \( \theta = \displaystyle\frac{ s }{ r } \)
    \\ \( \Delta \theta = \theta_f - \theta_i \)
    \\ \( \vec { \omega }_{\text{avg}} = \displaystyle\frac{ \Delta \theta }{ \Delta t } \)
    \\ \( \vec { \alpha }_{\text{avg}} = \displaystyle\frac{ \Delta \vec { \omega } }{ \Delta t } \)
\\ \textbf{Angular Velocity with Constant Acceleration:}
    \\ \( \vec{ \omega }_f = \vec{ \omega }_i + \vec{ \alpha } \Delta t \)
\\ \textbf{Angular Displacement with Constant Acceleration:}
    \\ \( \theta_f = \theta_i + \vec{ \omega }_i \Delta t + \displaystyle\frac{ \vec{ \alpha } \Delta t^2 }{ 2 } \)
\\ \textbf{Angular Velocity-Displacement Relation with \\ Constant Acceleration:}
    \\ \( \vec{ \omega }_{f}^2 = \vec{ \omega }_{i}^2 + 2 \vec{ \alpha } \Delta \theta \)
\\ \textbf{Angular to Linear Motion:}
    \\ \( \Delta x = r \Delta \theta \)
    \\ \( \vec{ v } = r \vec{ \omega } \)
    \\ \( \vec{ a }_T = r \vec{ \alpha } \)
\\ \textbf{Torque:}
    \\ \( \vec{ \tau } = r \vec{ F } \sin \theta \)
\\ \textbf{Archimedes's Law of Levers:}
    \\ \( \displaystyle\frac{ \vec{ F }_2 }{ \vec{ F }_1 } = \displaystyle\frac{ D_1 }{ D_2 } \)
\\ \textbf{Moment of Inertia:}
    \\ \( I = Cmr^2 \)
    \\ \( I_{\text{sys}} = \displaystyle\sum{ C_i m_i r_i^2 } \)
\\ \textbf{Parallel Axis Theorem:}
    \\ \( I' = I_{\text{cm}} + mx^2 \)
\\ \textbf{Newton's Second Law for Rotational Motion:}
    \\ \( \vec{ \alpha } = \displaystyle\frac{ \vec{ \tau }_{\text{net}} }{ I_{\text{sys}} } \)
\\ \textbf{Newton's Third Law for Rotational Motion:}
    \\ \( \Delta \vec{ L }_{a \rightarrow{b}} = - \Delta \vec{ L }_{b \rightarrow{a}} \)
\\ \textbf{Rotational Kinetic Energy:}
    \\ \( k_{\text{rot}} = \displaystyle\frac{ I \vec{ \omega }^2 }{ 2 } \)
\\ \textbf{Rotational Work:}
    \\ \( w = \vec{ \tau } \Delta \theta \)
\\ \textbf{Rotational Work-Energy Theorem:}
    \\ \( w = \Delta k_{\text{rot}} \)
\\ \textbf{Angular Momentum:}
    \\ \( \vec{ L } = I \vec{ \omega } \)
\\ \textbf{Orbital Angular Momentum:}
    \\ \( \vec{ L } = rm \vec{ v } \sin \theta \)
\\ \textbf{Angular Impulse-Momentum Theorem:}
    \\ \( \Delta \vec{ L } = I \Delta \vec{ \omega } = \vec{ \tau } \Delta t \)
\\ \textbf{Conservation of Angular Momentum:}
    \\ \( \vec{ L }_f - \vec{ L }_i = 0 \)

\subsection*{Fluid Mechanics}
\textbf{Fluid Pressure:}
    \\ \( P = P_{\text{atm}} + dgy \)
\\ \textbf{Buoyant Force:}
    \\ \( \vec{ F }_b = d_{\text{fluid}} V_{\text{disp}} g \)
\\ \textbf{Fluid Flow Rate:}
    \\ \( Q = \displaystyle\frac{ V }{ t } \)
\\ \textbf{Bernoulli's Equation: }
    \\ \( P_1 + dgy_1 + \displaystyle\frac{ d \vec{ v }_1^2 }{ 2 } = P_2 + dgy_2 + \displaystyle\frac{ d \vec{ v }_2^2 }{ 2 } \)
\\ \textbf{Torricelli's Theorem:}
    \\ \( \vec{ v }_2 = \sqrt{ 2g \Delta y } \)



\section*{\underline{Geometry}}
\textbf{Definition of Pi:}
    \\ \( \pi = \displaystyle\frac{ C }{ d } = \displaystyle\frac{ \tau }{ 2 } \)
\\ \textbf{Definition of Tau:}
    \\ \( \tau = \displaystyle\frac{ C }{ r } = 2 \pi \)

\subsection*{Two Dimensional}
\textbf{Pythagorean Theorem:}
    \\ \( a^2 + b^2 = c^2 \)
\\ \textbf{Angles of a Regular N-Gon:}
    \\ \( \theta_I = \displaystyle\frac{ 180^\circ ( n - 2 ) }{ n } = \displaystyle\frac{ \pi ( n - 2 ) }{ n } = \displaystyle\frac{ \tau ( n - 2 ) }{ 2n } \)
    \\ \( \theta_E = \displaystyle\frac{ 360^\circ }{ n } = \displaystyle\frac{ 2 \pi }{ n } = \displaystyle\frac{ \tau }{ n } \)
\\ \textbf{Euler's Formula:}
    \\ \( F + V = E + 2 \)
\\ \textbf{Area of a Triangle:}
    \\ \( A = \displaystyle\frac{ bh }{ 2 } \)
\\ \textbf{Alternative Area of a Triangle:}
    \\ \( A = \displaystyle\frac{ bc \sin A }{ 2 } \)
\\ \textbf{Heron's Formula:}
    \\ \( s = \displaystyle\frac{ a + b + c }{ 2 } \)
    \\ \( A = \sqrt{ s( s - a )( s - b )( s - c ) } \)
\\ \textbf{Area of a Parallelogram:}
    \\ \( A = bh \)
\\ \textbf{Area of a Square:}
    \\ \( A = s^2 \)
\\ \textbf{Area of a Trapezoid:}
    \\ \( A = \displaystyle\frac{ h( b_1 b_2 ) }{ 2 } \)
\\ \textbf{Area of a Rhombus:}
    \\ \( A = bh \)
\\ \textbf{Alternative Area of a Rhombus:}
    \\ \( A = \displaystyle\frac{ d_1 d_2 }{ 2 } \)
\\ \textbf{Area of a Kite:}
    \\ \( A = \displaystyle\frac{ d_1 d_2 }{ 2 } \)
\\ \textbf{Area of a Regular N-Gon:}
    \\ \( A = \displaystyle\frac{ ap }{ 2 } \)
\\ \textbf{Area of a Circle:}
    \\ \( A = \pi r^2 = \displaystyle\frac{ \tau r^2 }{ 2 } \)

\subsection*{Three Dimensional}
\textbf{Lateral and Surface Area of a Prism:}
    \\ \( A_L = ph \)
    \\ \( A_S = A_L + 2b \)
\\ \textbf{Lateral and Surface Area of a Cylinder: }
    \\ \( A_L = 2 \pi rh = \tau rh \)
    \\ \( A_S = A_L + 2 \pi r^2 = 2 \pi r( r + h ) = \tau r( r + h ) \)
\\ \textbf{Lateral and Surface Area of a Pyramid:}
    \\ \( A_L = \displaystyle\frac{ pl }{ 2 } \)
    \\ \( A_S = A_L + A_b \)
\\ \textbf{Lateral and Surface Area of a Cone:}
    \\ \( A_L = \pi rl = \displaystyle\frac{ \tau rl }{ 2 } \)
    \\ \( A_S = A_L + \pi r^2 = \pi r ( l + r ) \)
\\ \textbf{Surface Area of a Sphere:}
    \\ \( A_S = 4 \pi r^2 = 2 \tau r^2 \)
\\ \textbf{Volume of a Prism:}
    \\ \( V = bh \)
\\ \textbf{Volume of a Cylinder:}
    \\ \( V = \pi r^2 h = \displaystyle\frac{ \tau r^2 h }{ 2 } \)
\\ \textbf{Volume of a Pyramid:}
    \\ \( V = \displaystyle\frac{ bh }{ 3 } \)
\\ \textbf{Volume of a Cone:}
    \\ \( V = \displaystyle\frac{ \pi r^2 h }{ 3 } = \displaystyle\frac{ \tau r^2 h }{ 6 } \)
\\ \textbf{Volume of a Sphere:}
    \\ \( V = \displaystyle\frac{ 4 \pi r^3 }{ 3 } = \displaystyle\frac{ 2 \tau r^3 }{ 3 } \)



\section*{\underline{Trigonometry}}
\textbf{Definition of Sine:}
    \\ \( \sin A = \displaystyle\frac{ a }{ c } \)
\\ \textbf{Definition of Cosine:}
    \\ \( \cos A = \displaystyle\frac{ b }{ c } \)
\\ \textbf{Definition of Tangent:}
    \\ \( \tan A = \displaystyle\frac{ a }{ b } \)
\\ \textbf{Definition of Cosecant:}
    \\ \( \csc A = \displaystyle\frac{ c }{ a } \)
\\ \textbf{Definition of Secant:}
    \\ \( \sec A = \displaystyle\frac{ c }{ b } \)
\\ \textbf{Definition of Cotangent:}
    \\ \( \cot A = \displaystyle\frac{ b }{ a } \)
\\ \textbf{Law of Sines:}
    \\ \( \displaystyle\frac{ \sin A }{ a } = \displaystyle\frac{ \sin B }{ b } = \displaystyle\frac{ \sin C }{ c } \)
\\ \textbf{Law of Cosines:}
    \\ \( c^2 = a^2 + b^2 - 2ab \cos C \)
\\ \textbf{Law of Tangents:}
    \\ \( \displaystyle\frac{ \tan \frac{ A - B }{ 2 } }{ \tan \frac{ A + B }{ 2 } } = \displaystyle\frac{ a - b }{ a + b } \)
\\ \textbf{Law of Cotangents:}
    \\ \( s = \displaystyle\frac{ a + b + c }{ 2 } \)
    \\ \( \displaystyle\frac{ \cot \frac{ A }{ 2 } }{ s - a } = \displaystyle\frac{ \cot \frac{ B }{ 2 } }{ s - b } = \displaystyle\frac{ \cot \frac{ C }{ 2 } }{ s - c } \)
\\ \textbf{Pythagorean Identities:}
    \\ \( \sin^2 x + \cos^2 x = 1 \)
    \\ \( \tan^2 x + 1 = \sec^2 x \)
    \\ \( 1 + \cot^2 x = \csc^2 x \)
\\ \textbf{Sum/Difference Identities:}
    \\ \( \sin( x \pm y ) = \sin( x ) \cdot \cos( y ) \pm \cos( x ) \cdot \sin( y ) \)
    \\ \( \cos( x \pm y ) = \cos( x ) \cdot \cos( y ) \mp \sin( x ) \cdot \sin( y ) \)
    \\ \( \tan( x \pm y ) = \displaystyle\frac{ \tan( x ) \pm \tan( y ) }{ 1 \mp \tan( x ) \cdot \tan( y ) } \)
    \\ \( \csc( x \pm y ) = \displaystyle\frac{ \sec( x ) \sec( y ) \csc( x ) \csc( y ) }{ \sec( x ) \csc( y ) \pm \csc( x ) \sec( y ) } \)
    \\ \( \sec( x \pm y ) = \displaystyle\frac{ \sec( x ) \sec( y ) \csc( x ) \csc( y ) }{ \csc( x ) \csc( y ) \mp \sec( x ) \sec( y ) } \)
    \\ \( \cot( x \pm y ) = \displaystyle\frac{ \cot( x ) \cot( y ) \mp 1 }{ \cot( y ) \pm \cot( x ) } \)
\\ \textbf{Double Angle Identities:}
    \\ \( \sin( 2x ) = 2 \sin( x ) \cdot \cos( x ) \)
    \\ \( \cos( 2x ) = \cos^2( x ) - \sin^2( x ) \)
    \\ \( \tan( 2x ) = \displaystyle\frac{ 2 \tan( x ) }{ 1 - \tan^2( x ) } \)
    \\ \( \csc( 2x ) = \displaystyle\frac{ \sec( x ) \cdot \csc( x ) }{ 2 } \)
    \\ \( \sec( 2x ) = \displaystyle\frac{ \sec^2( x ) }{ 2 - \sec^2( x ) } \)
    \\ \( \cot( 2x ) = \displaystyle\frac{ \cot^2( x ) - 1 }{ 2 \cot( x ) } \)
\\ \textbf{Half Angle Identities:}
    \\ \( \sin \left( \displaystyle\frac{ x }{ 2 } \right) = \text{sgn} \left( \sin \left( \displaystyle\frac{ x }{ 2 } \right) \right) \sqrt{ \displaystyle\frac{ 1 - \cos( x ) }{ 2 } } \)
    \\ \( \cos \left( \displaystyle\frac{ x }{ 2 } \right) = \text{sgn} \left( \cos \left( \displaystyle\frac{ x }{ 2 } \right) \right) \sqrt{ \displaystyle\frac{ 1 + \cos( x ) }{ 2 } } \)
    \\ \( \tan \left( \displaystyle\frac{ x }{ 2 } \right) = \displaystyle\frac{ \sin( x ) }{ 1 + \cos( x ) } \)
    \\ \( \csc \left( \displaystyle\frac{ x }{ 2 } \right) = \text{sgn} \left( \sin \left( \displaystyle\frac{ x }{ 2 } \right) \right) \sqrt{ \displaystyle\frac{ 2 }{ 1 - \cos( x ) } } \)
    \\ \( \sec \left( \displaystyle\frac{ x }{ 2 } \right) = \text{sgn} \left( \cos \left( \displaystyle\frac{ x }{ 2 } \right) \right) \sqrt{ \displaystyle\frac{ 2 }{ 1 + \cos( x ) } } \)
    \\ \( \cot \left( \displaystyle\frac{ x }{ 2 } \right) = \displaystyle\frac{ 1 + \cos( x ) }{ \sin( x ) } \)
\\ \textbf{Product to Sum Identities:}
    \\ \( \sin( x ) \cdot \sin( y ) = \displaystyle\frac{ \cos( x - y ) - \cos( x + y ) }{ 2 } \)
    \\ \( \cos( x ) \cdot \cos( y ) = \displaystyle\frac{ \cos( x - y ) + \cos( x + y ) }{ 2 } \)
    \\ \( \sin( x ) \cdot \cos( y ) = \displaystyle\frac{ \sin( x - y ) - \sin( x + y ) }{ 2 } \)
    \\ \( \tan( x ) \cdot \tan( y ) = \displaystyle\frac{ \cos( x - y ) - \cos( x + y ) }{ \cos( x - y ) + \cos( x + y ) } \)
    \\ \( \tan( x ) \cdot \cot( y ) = \frac{ \cos( x - y ) + \cos( x + y ) }{ \cos( x - y ) - \cos( x + y ) } \)
\\ \textbf{Sum to Product Identities:}
    \\ \( \sin( x ) \pm \sin( y ) = 2 \sin \left( \displaystyle\frac{ x \pm y }{ 2 } \right) \cos \left( \displaystyle\frac{ x \mp y }{ 2 } \right) \)
    \\ \( \cos( x ) + \cos( y ) = 2 \cos \left( \displaystyle\frac{ x + y }{ 2 } \right) \cos \left( \displaystyle\frac{ x - y }{ 2 } \right) \)
    \\ \( \cos( x ) - \cos( y ) = -2 \sin \left( \displaystyle\frac{ x + y }{ 2 } \right) \sin \left( \displaystyle\frac{ x - y }{ 2 } \right) \)
    \\ \( \tan( x ) \pm \tan( y ) = \frac{ \sin( x \pm y ) }{ \cos( x ) \cdot \cos( y ) } \)
\\ \textbf{Polar Coordinate Equations:}
    \\ \( x^2 + y^2 = r^2 \)
    \\ \( \tan \theta = \displaystyle\frac{ y }{ x } \)
    \\ \( x = r \cos \theta \)
    \\ \( y = r \sin \theta \)
\\ \textbf{Polar Form of Conic Sections:}
    \\ \( r = \displaystyle\frac{ de }{ 1 + e \sin \theta } \)
    \\ \( r = \displaystyle\frac{ de }{ 1 + e \cos \theta } \)



\section*{\underline{Hyperbolic Trig}}
\textbf{Definition of Hyperbolic Sine:}
    \\ \( \sinh x = \displaystyle\frac{ e^x - e^{-x} }{ 2 } = -i \sin ix \)
\\ \textbf{Definition of Hyperbolic Cosine:}
    \\ \( \cosh x = \displaystyle\frac{ e^x + e^{-x} }{ 2 } = \cos ix \)
\\ \textbf{Definition of Hyperbolic Tangent:}
    \\ \( \tanh x = \displaystyle\frac{ e^x - e^{-x} }{ e^x + e^{-x} } = -i \tan ix \)
\\ \textbf{Definition of Hyperbolic Cosecant:}
    \\ \( \text{csch } x = \displaystyle\frac{ 2 }{ e^x - e^{-x} } = i \csc ix \)
\\ \textbf{Definition of Hyperbolic Secant:}
    \\ \( \text{sech } x = \displaystyle\frac{ 2 }{ e^x + e^{-x} } = \sec ix \)
\\ \textbf{Definition of Hyperbolic Cotangent:}
    \\ \( \coth x = \displaystyle\frac{ e^x + e^{-x} }{ e^x - e^{-x} } = i \cot ix \)
\\ \textbf{Hyperbolic Pythagorean Identities:}
    \\ \( \sinh^2 x - \cosh^2 x = 1 \)
    \\ \( 1 - \tanh^2 x = \text{sech}^2 \text{ } x \)
    \\ \( \coth^2 x - 1 = \text{csch}^2 \text{ } x \)
\\ \textbf{Hyperbolic Sum/Difference Identities:}
    \\ \( \sinh( x \pm y ) = \sinh( x ) \cdot \cosh( y ) \pm \cosh( x ) \cdot \sinh( y ) \)
    \\ \( \cosh( x \pm y ) = \cosh( x ) \cdot \cosh( y ) \pm \sinh( x ) \sinh( y ) \)
    \\ \( \tanh( x \pm y ) = \displaystyle\frac{ \tanh( x ) \pm \tanh( y ) }{ 1 \pm \tanh( x ) \cdot \tanh( y ) } \)
\\ \textbf{Hyperbolic Double Angle Identities:}
    \\ \( \sinh( 2x ) = 2 \sinh( x ) \cdot \cosh( x ) \)
    \\ \( \cosh( 2x ) = \sinh^2( x ) + \cosh^2( x ) \)
    \\ \( \tanh( 2x ) = \displaystyle\frac{ 2 \tanh( x ) }{ 1 + \tanh^2( x ) } \)
\\ \textbf{Hyperbolic Half Angle Identities:}
    \\ \( \sinh \left( \displaystyle\frac{ x }{ 2 } \right) = \text{sgn}( x ) \sqrt{ \displaystyle\frac{ \cosh( x ) - 1 }{ 2 } } \)
    \\ \( \cosh \left( \displaystyle\frac{ x }{ 2 } \right) = \sqrt{ \displaystyle\frac{ \cosh( x ) + 1 }{ 2 } } \)
    \\ \( \tanh \left( \displaystyle\frac{ x }{ 2 } \right) = \displaystyle\frac{ \sinh( x ) }{ \cosh( x ) + 1 } \)
\\ \textbf{Hyperbolic Sum to Product Identities:}
    \\ \( \sinh( x ) + \sinh( y ) = 2 \sinh \left( \displaystyle\frac{ x + y }{ 2 } \right) \cosh \left( \displaystyle\frac{ x - y }{ 2 } \right) \)
    \\ \( \sinh( x ) - \sinh( y ) = 2 \cosh \left( \displaystyle\frac{ x + y }{ 2 } \right) \sinh \left( \displaystyle\frac{ x - y }{ 2 } \right) \)
    \\ \( \cosh( x ) + \cosh( y ) = 2 \cosh \left( \displaystyle\frac{ x + y }{ 2 } \right) \cosh \left( \displaystyle\frac{ x - y }{ 2 } \right) \)
    \\ \( \cosh( x ) - \cosh( y ) = -2 \sinh \left( \displaystyle\frac{ x + y }{ 2 } \right) \sinh \left( \displaystyle\frac{ x - y }{ 2 } \right) \)



\section*{\underline{Algebra}}
\textbf{Fundamental Theorem of Algebra:}
    \\ \( \forall f : [ f( x ) = ax^n + bx^{n - 1} + ... + zx + c ] \) 
     \\ \( \exists [ \chi_i \in \mathbb{ C } : f( \chi_i ) = 0 ] ; i \in \mathbb{ Z }_{1 \le i \le n} \)
\\ \textbf{Definition of the Imaginary Unit:}
    \\ \( i^2 = -1 \)

\subsection*{Formulae}
\textbf{Distance Formula:}
    \\ \( D = \sqrt{ x^2 + y^2 } \)
\\ \textbf{Slope Formula:}
    \\ \( m = \displaystyle\frac{ \Delta x }{ \Delta y } = \displaystyle\frac{ x_2 - x_1 }{ y_2 - y_1 } \)
\\ \textbf{Discriminant of a Quadratic Equation:}
    \\ \( \Delta = b^2 - 4ac \)
\\ \textbf{Quadratic Formula:}
    \\ \( x = \displaystyle\frac{ -b \pm \sqrt{ b^2 - 4ac } }{ 2a } \)
\\ \textbf{Cubic Formula:}
    \\ \( \xi = \displaystyle\frac{ -1 + \sqrt{ -3 } }{ 2 } \)
    \\ \( \Delta_0 = b^2 - 3ac  \)
    \\ \( \Delta_1 = 2b^2 - 9abc + 27 a^2 d \)
    \\ \( C = \sqrt[3]{ \displaystyle\frac{ \Delta_1 \pm \sqrt{ \Delta_1^2 } -4 \Delta_0^3 }{ 2 } } \)
    \\ \( x_k = \displaystyle\frac{ -1 }{ 3a } \left(  b + \zeta^kC + \displaystyle\frac{ \Delta_0 }{ \zeta^k C } \right) ; k \in \{ 0, 1, 2 \} \)
\\ \textbf{Binomial Formula:}
    \\ \( ( a + b )^n = \displaystyle\sum_{k = 0}^{n} { \binom{n}{k} a^{n - k} b^k } \)

\subsection*{Functions}
\textbf{Slope-Intercept Form of a Line:}
    \\ \( y = mx + b \)
\\ \textbf{Point-Slope Form of a Line:}
    \\ \( y - y_1 = m( x - x_1 ) \)
\\ \textbf{Standard Form of a Line:}
    \\ \( Ax + By = C \)
\\ \textbf{Exponential Function:}
    \\ \( y = a \cdot b^x \)
\\ \textbf{Vertex Form of a Quadratic:}
    \\ \( y=  a( x - h )^2 + k \)
\\ \textbf{Factored Form of a Quadratic:}
    \\ \( y = ( x - r_1 ) ( x - r_2 ) \)
\\ \textbf{Standard Form of a Quadratic:}
    \\ \( y = ax^2 +bx + c \)
\\ \textbf{Standard Form of a Polynomial:}
    \\ \( ax^n + bx^{n-1} + cx^{n-2}... + vx^2 + wx + z = 0 \)
\\ \textbf{Standard Form of an Absolute Value Function:}
    \\ \( y = a| x - h | + k \)
\\ \textbf{Standard Form of a Square Root Function:}
    \\ \( y = a \sqrt{ x - h } + k \)

\subsection*{Conic Sections}
\textbf{Equation of a Circle:}
    \\ \( x^2 + y^2 = r^2 \)
\\ \textbf{Equation of a Horizontal Ellipse:}
    \\ \( \displaystyle\frac{ ( x - h )^2 }{ a^2 } + \displaystyle\frac{ ( y - k )^2 }{ b^2 } = 1 \)
\\ \textbf{Equation of a Vertical Ellipse:}
    \\ \( \displaystyle\frac{ ( x - h )^2 }{ b^2 } + \displaystyle\frac{ ( y - k )^2 }{ a^2 } = 1 \)
\\ \textbf{Equation of a Horizontal Hyperbola:}
    \\ \( \displaystyle\frac{ ( x - h )^2 }{ a^2 } - \displaystyle\frac{ ( y - k )^2 }{ b^2 } = 1 \)
\\ \textbf{Equation of a Vertical Hyperbola:}
    \\ \( \displaystyle\frac{ ( y - k )^2 }{ a^2 } - \displaystyle\frac{ ( x - h )^2 }{ b^2 } = 1 \)
\\ \textbf{Equation of a Horizontal Parabola:}
    \\ \( x = \displaystyle\frac{ 1 }{ 4p } (y - k)^2 +h \)
\\ \textbf{Equation of a Vertical Parabola:}
    \\ \( y = \displaystyle\frac{ 1 }{ 4p } (x - h)^2 +k \)
\\ \textbf{Standard Form of a Conic Section:}
    \\ \( ax^2 + bxy + cy^2 + dx + ey + f = 0 \)

\subsection*{Sequences and Series}
\textbf{Recursive Arithmetic Sequence:}
    \\ \( a_1 = x \)
    \\ \( a_n = a_{n - 1} + d \)
\\ \textbf{Explicit Arithmetic Sequence:}
    \\ \( a_n = d( n - 1 ) + a_1 \)
\\ \textbf{Recursive Geometric Sequence:}
    \\ \( a_1 = x \)
    \\ \( a_n = a_{n - 1} \cdot r \)
\\ \textbf{Explicit Geometric Sequence:}
    \\ \( a_n = a_1 ( r )^{n - 1} \)
\\ \textbf{Sum of an Arithmetic Series when the Last Term is Given:}
    \\ \( S_n = \displaystyle\frac{ n( a_1 + a_n ) }{ 2 } \)
\\ \textbf{Sum of an Arithmetic Series when the Last Term is Not Given:}
    \\ \( S_n = \displaystyle\frac{ n( 2a_1 + d( n - 1 ) ) }{ 2 } \)
\\ \textbf{Sum of a Geometric Series:}
    \\ \( S_n = \displaystyle\frac{ a_1 ( 1 - r^n ) }{ 1 - r } \)
\\ \textbf{Sum of an Infinite Geometric Series:}
    \\ \( S = \displaystyle\frac{ a_1 }{ 1 - r } ; 0 < | x | < 1 \)

\subsection*{Methods}
\textbf{Difference of Squares:}
    \\ \( a^2 - b^2 = ( a + b ) ( a - b ) \)
\\ \textbf{Sum of Squares:}
    \\ \( a^2 + b^2 = ( a + bi ) ( a - bi ) \)
\\ \textbf{Difference of Cubes:}
    \\ \( a^3 - b^3 = ( a - b )( a^2 + ab + b^2 ) \)
\\ \textbf{Sum of Cubes:}
    \\ \( a^3 + b^3 = ( a + b )( a^2 - ab + b^2 ) \)
\\ \textbf{Various Summation of Single Terms:}
    \\ \( \displaystyle\sum_{k = 1}^n { k } = \displaystyle\frac{ n( n + 1 ) }{ 2 } \)
    \\ \( \displaystyle\sum_{k = 1}^n { k^2 } = \displaystyle\frac{ n( n + 1 )( 2n + 1 ) }{ 6 } \)
    \\ \( \displaystyle\sum_{k = 1}^n { k^3 } = \displaystyle\frac{ n^2( n + 1 )^2 }{ 4 } \)
    \\ \( \displaystyle\sum_{k = 1}^n { k^4 } = \displaystyle\frac{ n( 2n + 1 )( n + 1 )( 3n^2 + 3n - 1 ) }{ 30 } \)
    \\ \( \displaystyle\sum_{k = 1}^n { k^5 } = \displaystyle\frac{ n^2 ( 2n^2 + 2n - 1 )( n + 1 )^2 }{ 12 } \)
    \\ \( \displaystyle\sum_{k = 1}^n { k^6 } = \displaystyle\frac{ n( 2n + 1 )( n + 1 )( 3n^4 + 6n^3 - 3n + 1 ) }{ 42 } \)
    \\ \( \displaystyle\sum_{k = 1}^n { k^7 } = \displaystyle\frac{ n^2 ( 3n^4 + 6n^3 - n^2 - 4n + 2 )( n + 1 )^2 }{ 24 } \)



\section*{\underline{Statistics}}
\textbf{Calculation of the Mean Average:}
    \\ \( \bar{ x } = \displaystyle\frac{ \displaystyle\sum{ x_i } }{ n } \)
\\ \textbf{Calculation of Sample Standard Distribution:}
    \\ \( S_x = \sqrt{ \frac{ \displaystyle\sum{ ( x_i - \bar{ x } ) } }{ n - 1 } } \)

\subsection*{Probability}
\textbf{Probability Laws:}
    \\ \( P( \lnot A ) = 1 - P( A ) \)
    \\ \( P( A \lor B ) = P( A ) + P( B ) - P( A \land B ) \)
    \\ \( P( A \land B ) = P( A ) \cdot P( B | A ) \)
    \\ \( P( A | B ) = \displaystyle\frac{ P( A \land B ) }{ P( B ) } \)
\\ \textbf{Bayes's Theorem:}
    \\ \( P( A | B ) = \displaystyle\frac{ P( B | A ) \cdot P( A ) }{ P( B ) } \)



\section*{\underline{Combinatorics}}
\textbf{Number of Permutations of a Given System:}
    \\ \( _nP_r = \displaystyle\frac{ n! }{ ( n - r )! } \)
\\ \textbf{Number of Combinations of a Given System:}
    \\ \( _nC_r = \left( \displaystyle\substack{ n \\ r } \right) = \displaystyle\frac{ n! }{ r! ( n - 1 )! } \)
\\ \textbf{Permutations of Repeated Objects:}
    \\ \( _nP_n = \displaystyle\frac{ n! }{ a! \cdot b! \cdot c! ... } \)
\\ \textbf{Circular Permutations:}
    \\ \( _nP_n = ( n - 1 )! \)
\\ \textbf{Circular Permutations with Repeated Objects:}
    \\ \( _nP_n = \displaystyle\frac{ ( n - 1 )! }{ a! \cdot b! \cdot c! ... } \)
\\ \textbf{Circular Permutations of Flipable Circles:}
    \\ \( _nP_n = \displaystyle\frac{ ( n - 1 )! }{ 2 } \)



\section*{\underline{Calculus}}
\textbf{Fundamental Theorem of Calculus, Part One:}
    \\ \( f( x ) = \displaystyle\int_{0}^{h( x )} { g( t ) dt } \)
    \\ \( \implies \displaystyle\frac{ d }{ dx } f( x ) = ( g \circ h )( x ) \cdot \displaystyle\frac{ d }{ dx } h( x ) \)
\\ \textbf{Fundamental Theorem of Calculus, Part Two:}
    \\ \( \displaystyle\int_{a}^{b} { \displaystyle\frac{ d }{ dx } f( x ) dx } = f( b ) - f( a ) \)

\subsection*{Limit}
\textbf{L'\^{o}pital's Rule:}
    \\ \( \displaystyle\lim_{x \rightarrow{c}} { \displaystyle\frac{ f( x ) }{ g( x ) } } = \displaystyle\lim_{x \rightarrow{c}} { \displaystyle\frac{ \frac{ d }{ dx } f( x ) }{ \frac{ d }{ dx } g( x ) } }  \)
    \\ \( [ | f( c) | = | g( c ) | = \infty] \lor [ | f( c) | = | g( c ) | = 0] \)
\\ \textbf{Squeeze Theorem:}
    \\ \( \forall f( x ) : [ g( x )\le f( x ) \le h( x ) \forall x \in [ b, c ] ] \)
    \\ \( \exists [ \displaystyle\lim_{x \rightarrow{a}} { f( x ) } ] = \displaystyle\lim_{x \rightarrow{a}} { g( x ) } = \displaystyle\lim_{x \rightarrow{a}} { h( x ) } \)
    \\ \( \iff \exists [ \displaystyle\lim_{x \rightarrow{a}} { g( x ) } = \displaystyle\lim_{x \rightarrow{a}} { h( x ) } ] \land [a \in [ b, c ]] \)
\\ \textbf{Common Limits: }
    \\ \( \displaystyle\lim_{x \rightarrow{0} } { \displaystyle\frac{ e^x - 1 }{ x } } = 1 \)
    \\ \( \displaystyle\lim_{x \rightarrow{\infty}} { \left( 1 + \displaystyle\frac{ x }{ n } \right)^n } = e^x \)
    \\ \( \displaystyle\lim_{x \rightarrow{\infty}} { \displaystyle\frac{ \ln x }{ x } } = 0 \)
\\ \textbf{Common Trigonometric Limits:}
    \\ \( \displaystyle\lim_{x \rightarrow{0}} { \displaystyle\frac{ \sin x }{ x } } = 1 \)
    \\ \( \displaystyle\lim_{x \rightarrow{\infty}} { \displaystyle\frac{ \sin x }{ x } } = 0 \)
    \\ \( \displaystyle\lim_{x \rightarrow{0}} { \displaystyle\frac{ -1 + \cos x }{ x } } = 0 \)
    \\ \( \displaystyle\lim_{x \rightarrow{\infty}} { \displaystyle\frac{ -1 + \cos x }{ x } } = 0 \)
    \\ \( \displaystyle\lim_{x \rightarrow{\infty}} { \displaystyle\frac{ 1 + \cos x }{ x } } = 0 \)
    \\ \( \displaystyle\lim_{x \rightarrow{0}} { \displaystyle\frac{ \tan x }{ x } } = 1 \)

\subsection*{Differential}
\textbf{Definition of the Derivative:}
    \\ \( \displaystyle\frac{ d }{ dx } f( x ) = \displaystyle\lim_{h \rightarrow{0}} { \displaystyle\frac{ f( x + h ) - f( h ) }{ h } } = \displaystyle\lim_{x \rightarrow{a}} { \displaystyle\frac{ f( x ) - f( a ) }{ x - a } } \)
\\ \textbf{Mean Value Theorem for Derivatives:}
    \\ \( \forall f \exists c \in [a, b] : \displaystyle\frac{ f( b ) - f( a ) }{ b - a } = \displaystyle\frac{ d }{ dx } f( c ) \)
\\ \textbf{Definition of the Taylor Series:}
    \\ \( f_a( x ) = \displaystyle\sum_{n = 0}^{\infty} { \displaystyle\frac{ \frac{ d }{ dx } f( a ) }{ n! } ( x - a )^n } \)
\\ \textbf{Newton's Method:}
    \\ \( x_1 = a \)
    \\ \( x_n = \displaystyle\frac{ -f( x_{n - 1} ) }{ \frac{ d }{ dx } f( x_{n - 1} ) } + x_{n - 1} \)
\\ \textbf{Trigonometric Derivatives:}
    \\ \( \displaystyle\frac{ d }{ dx } \sin x = \cos x \)
    \\ \( \displaystyle\frac{ d }{ dx } \cos x = - \sin x \)
    \\ \( \displaystyle\frac{ d }{ dx } \tan x = \sec^2 x \)
    \\ \( \displaystyle\frac{ d }{ dx } \csc x = - \csc x \cdot \cot x \)
    \\ \( \displaystyle\frac{ d }{ dx } \sec x = \sec x \cdot \tan x \)
    \\ \( \displaystyle\frac{ d }{ dx } \cot x = - \csc^2 x \)
\\ \textbf{Hyperbolic Trigonometric Derivatives:}
    \\ \( \displaystyle\frac{ d }{ dx } \sinh x = \cosh x \)
    \\ \( \displaystyle\frac{ d }{ dx } \cosh x = \sinh x \)
    \\ \( \displaystyle\frac{ d }{ dx } \tanh x = \text{sech}^2 x \)
    \\ \( \displaystyle\frac{ d }{ dx } \text{csch } x = - \text{csch } x \cdot \coth x \)
    \\ \( \displaystyle\frac{ d }{ dx } \text{sech } x = - \text{sech } x \cdot \tanh x \)
    \\ \( \displaystyle\frac{ d }{ dx } \coth x = - \text{csch}^2 x \)
\\ \textbf{Inverse Trigonometric Derivatives:}
    \\ \( \displaystyle\frac{ d }{ dx } \arcsin x = \displaystyle\frac{ 1 }{ \sqrt{ 1 - x^2 } } \)
    \\ \( \displaystyle\frac{ d }{ dx } \arccos x = \displaystyle\frac{ -1 }{ \sqrt{ 1 - x^2 } } \)
    \\ \( \displaystyle\frac{ d }{ dx } \arctan x = \displaystyle\frac{ 1 }{ x^2 + 1 } \)
    \\ \( \displaystyle\frac{ d }{ dx } \text{arccsc } x = \displaystyle\frac{ 1 }{ | x | \sqrt{ x^2 - 1 } } \)
    \\ \( \displaystyle\frac{ d }{ dx } \text{arcsec } x = \displaystyle\frac{ -1 }{ | x | \sqrt{ x^2 - 1 } } \)
    \\ \( \displaystyle\frac{ d }{ dx } \text{arccot } x = \displaystyle\frac{ -1 }{ x^2 + 1 } \)
\\ \textbf{Inverse Hyperbolic Trigonometric Derivatives:}
    \\ \( \displaystyle\frac{ d }{ dx } \text{arsinh } x = \displaystyle\frac{ 1 }{ \sqrt{ x^2 + 1 } } \)
    \\ \( \displaystyle\frac{ d }{ dx } \text{arcosh } x = \displaystyle\frac{ 1 }{ \sqrt{ x^2 + 1 } } ; x > 1 \)
    \\ \( \displaystyle\frac{ d }{ dx } \text{arcoth } x = \displaystyle\frac{ 1 }{ 1 - x^2 } ; | x | < 1 \)
    \\ \( \displaystyle\frac{ d }{ dx } \text{arcsch } x = \displaystyle\frac{ -1 }{ | x | \sqrt{ 1 + x^2 } } ; x \neq 0 \)
    \\ \( \displaystyle\frac{ d }{ dx } \text{arsech } x = \displaystyle\frac{ -1 }{ x \sqrt{ 1 + x^2 } } ; 0 < x < 1 \)
    \\ \( \displaystyle\frac{ d }{ dx } \text{arcoth } x = \displaystyle\frac{ 1 }{ 1 - x^2 } ; | x | > 1 \)
    
\subsection*{Integral}
\textbf{Definition of the Riemann Integral:}
    \\ \( \displaystyle\int_{a}^{b} { f( x ) dx } = \displaystyle\lim_{n \rightarrow{\infty}} \displaystyle\sum_{i = 1}^{n} { f( x_{i}^{*} ) \Delta x ; \Delta x = \displaystyle\frac{ b - a }{ n } } \)
\\ \textbf{Mean Value Theorem for Integrals:}
    \\ \( \forall f \exists c \in [a, b] : \displaystyle\frac{ \displaystyle\int_{a}^{b} { f( x ) dx } }{ b - a } = f( c ) \)
\\ \textbf{Trigonometric Integrals:}
    \\ \( \displaystyle\int { \sin x dx } = - \cos x + C \)
    \\ \( \displaystyle\int { \cos x dx } = \sin x + C \)
    \\ \( \displaystyle\int { \tan x dx } = \ln| \sec x | + C \)
    \\ \( \displaystyle\int { \csc x dx } = \ln| \csc x - \cot x | + C \)
    \\ \( \displaystyle\int { \sec x dx } = \ln| \sec x + \tan x | + C \)
    \\ \( \displaystyle\int { \cot x dx } = - \ln| \csc x | + C \)
\\ \textbf{Hyperbolic Trigonometric Integrals:}
    \\ \( \displaystyle\int { \sinh x dx } = \cosh x + C \)
    \\ \( \displaystyle\int { \cosh x dx } = \sinh x + C \)
    \\ \( \displaystyle\int { \tanh x dx } = \ln \cosh x + C \)
    \\ \( \displaystyle\int { \text{csch } x dx } = \ln \left| \tanh \displaystyle\frac{ x }{ 2 } \right| + C \)
    \\ \( \displaystyle\int { \text{sech } x dx } = \arctan \sinh x + C \)
    \\ \( \displaystyle\int { \coth x dx } = \ln| \sinh x | + C \)
\\ \textbf{Inverse Trigonometric Integrals: }
    \\ \( \displaystyle\int { \arcsin x dx } = x \arcsin x + \sqrt{ 1 - x^2 } + C ; | x | \le 1 \)
    \\ \( \displaystyle\int { \arccos x dx } = x \arccos x - \sqrt{ 1 - x^2 } + C ; | x | \le 1 \)
    \\ \( \displaystyle\int { \arctan x dx } = x \arctan x - \displaystyle\frac{ \ln| 1 + x^2 | }{ 2 } + C \)
    \\ \( \displaystyle\int { \text{arccsc } x dx } = x \text{ arccsc } x + \ln \left| x \left( 1 + \sqrt{ 1 - x^{-2} } \right) \right| + C ; | x | \ge 1 \)
    \\ \( \displaystyle\int { \text{arcsec } x dx } = x \text{ arcsec } x - \ln \left| x \left( 1 + \sqrt{ 1 - x^{-2} } \right) \right| + C ; | x | \ge 1 \)
    \\ \( \displaystyle\int { \text{arccot } x dx } = x \text{ arccot } x + \displaystyle\frac{ \ln| 1 + x^2 | }{ 2 } + C \)

\subsection*{Implementations}
\textbf{Arc Length:}
    \\ \( \left. L_{f( x )} \right|_{a}^{b} = \displaystyle\int_{a}^{b} { \sqrt{ 1 + \left( \displaystyle\frac{ d }{ dx } f( x ) \right)^2 } dx } \)
\\ \textbf{Surface Area of a Rotational Solid:}
    \\ \( \left. S_{f( x )} \right|_{a}^{b} = 2 \pi \displaystyle\int_{a}^{b} { r \cdot \sqrt{ 1 + \left( \displaystyle\frac{ d }{ dx } f( x ) \right)^2 } dx } \)
\\ \textbf{Volume of a Rotational Solid:}
    \\ \( \left. V_{f( x )} \right|_{a}^{b} = 2 \pi \displaystyle\int_{a}^{b} { f( x )^2 dx } \)
\\ \textbf{Definition of Euler's Number:}
    \\ \( e = \displaystyle\lim_{n \rightarrow{\infty}} { \left( 1 + \displaystyle\frac{ 1 }{ n } \right)^n } \)
\\ \textbf{Exponential McLauren Series:}
    \\ \( e^x = \displaystyle\sum_{n = 0}^{\infty} { \displaystyle\frac{ x^n }{ n! } } \)
\\ \textbf{Euler's Formula:}
    \\ \( e^{i\theta} = \cos{\theta} + i\sin{\theta} \)
\\ \textbf{Euler's Identity:}
    \\ \( e^{i \pi} + 1 = 0 \)
    \\ \( e^{i\tau} = 1  \)
\\ \textbf{Power Series of Trigonometric Functions: }
    \\ \( \sin x = \displaystyle\sum_{n = 0}^{\infty} { \displaystyle\frac{ ( -1 )^n }{ ( 2n + 1 )! } \cdot x^{2n + 1} } \)
    \\ \( \cos x = \displaystyle\sum_{n = 0}^{\infty} { \displaystyle\frac{ ( -1 )^n }{ (2n)! } \cdot x^{2n} } \)
    \\ \( \tan x = \displaystyle\sum_{n = 0}^{\infty} { \displaystyle\frac{ U_{2n + 1} }{ ( 2n + 1 )! } \cdot x^{2n + 1} } ;  | x | < \displaystyle\frac{ \tau }{ 4 } \)
    \\ \( \csc x = \displaystyle\sum_{n = 0}^{\infty} { \displaystyle\frac{ ( -1 )^n 2( 2^{2n - 1} - 1 ) B_{2n} }{ ( 2n + 1 )! } \cdot x^{2n} } ; 0 < x < \displaystyle\frac{ \tau }{ 2 } \)
    \\ \( \sec x = \displaystyle\sum_{n = 0}^{\infty} { \displaystyle\frac{ U_{2n} }{ ( 2n )! } \cdot x^{2n} } ; | x | < \displaystyle\frac{ \tau }{ 4 } \)
    \\ \( \cot x = \displaystyle\sum_{n = 0}^{\infty} { \displaystyle\frac{ ( -1 )^n 2^{2n - 1} B_{2n} }{ ( 2n + 1 )! } \cdot x^{2n + 1} } ; 0 < x < \displaystyle\frac{ \tau }{ 2 } \)



\section*{\underline{Analysis}}
\textbf{Definition of the Gamma Function:}
    \\ \( \Gamma( z ) = ( z - 1 )! = \displaystyle\int_{0}^{\infty} { x^{z - 1} e^{-x} dx } \)
\\ \textbf{Definition of the Riemann Zeta Function:}
    \\ \( \zeta( s ) = \displaystyle\sum_{n = 1}^{\infty} { \displaystyle\frac{ 1 }{ n^s } } \)
\\ \textbf{Definition of the Fourier Transform:}
    \\ \( \hat{g}( f ) = \displaystyle\int_{-\infty}^{\infty} { g( t ) e^{-\tau i ft} dt } \)
\\ \textbf{Definition of the Laplace Transform:}
    \\ \( \mathcal{L} \{ f \} ( s ) = \displaystyle\int_{0}^{\infty} { f( t ) e^{-st} dt } \)



\section*{\underline{Number Theory}}
\textbf{Fermat’s Last Theorem:}
    \\ \( \lnot [ \forall n \in \mathbb{Z} : [ n \ge 3 ] \exists [a, b, c \in \mathbb{Z} ] : [ a^n +b^n = c^n ] ] \)
\\ \textbf{Fermat’s Little Theorem:}
    \\ \( \forall [ a \in \mathbb{Z} ] \land [ n \in \mathbb{P} ] [ ( a^n - a ) \bmod n = 0 ] \)
\\ \textbf{Wilson’s Theorem:}
    \\ \( [ ( ( n - 1 )! - 1 ) \bmod n = 0 ] \iff [ n \in \mathbb{P}] \)
\\ \textbf{The Fibonacci Sequence:}
    \\ \( F_n = F_{n - 1} + F_{n - 2} ; F_0 = 0, F_1 = 1 \)
\\ \textbf{Definition of the Golden Ratio:}
    \\ \( \phi = \displaystyle\frac{ 1 + \sqrt{ 5 } }{ 2 } = 1 + \displaystyle\frac{ 1 }{ 1 + \displaystyle\frac{ 1 }{ 1 + \displaystyle\frac{ 1 }{ 1 + \displaystyle\frac{ 1 }{ 1 + ... } } } } \)
\\ \textbf{Pythagorean Triples:}
    \\ \( \left. \substack{ a = m^2 - n^2 \\ b = 2mn \\ c = m^2 + n^2 } \right\} ( m > n ) \land ( m, n \in \mathbb{Z} ) \)



\section*{\underline{Economics}}
\textbf{Simple Interest:}
    \\ \( A = P( 1 + rt ) \)
\\ \textbf{Compound Interest:}
    \\ \( A = P( 1 + \displaystyle\frac{ 1 }{ n } )^{nt} \)
\\ \textbf{Continual Compound Interest:}
    \\ \( A = P e^{rt} \)
\\ \textbf{Future Value of an Ordinary Annuity:}
    \\ \( A = \displaystyle\frac{ Pn \left[ \left( 1 + \frac{ r }{ n } \right)^{nt} - 1 \right] }{ r } \)



\section*{\underline{Electrical Engineering}}
\textbf{Ohm's Law:}
    \\ \( V = IR \)



\section*{\underline{Cryptology}}
\textbf{Caesar Cipher:}
    \\ \( y = ( x + n ) \bmod 26 \)
\\ \textbf{Affine Cipher:}
    \\ \( y = ( ax + b ) \bmod 26 \)
\\ \textbf{Password Entropy:}
    \\ \( E = L \cdot \log_2{ R } \)



\section*{\underline{Medical Science}}
\textbf{Glaister Equation:}
    \\ \( \Delta t_{death} \approx \displaystyle\frac{ T_{baseline} - T_{rectal} }{ R_{cooling} } \)

\end{document}