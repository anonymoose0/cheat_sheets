\begin{minipage}{\textwidth}
\textbf{Definitions of Angular Displacement, \\Velocity, and Acceleration:}
    \\ \( \theta = \displaystyle\frac{ s }{ r } \)
    \\ \( \Delta \theta = \theta_f - \theta_i \)
    \\ \( \vec { \omega }_{\text{avg}} = \displaystyle\frac{ \Delta \theta }{ \Delta t } \)
    \\ \( \vec { \alpha }_{\text{avg}} = \displaystyle\frac{ \Delta \vec { \omega } }{ \Delta t } \)
\end{minipage}
\\ \begin{minipage}{\textwidth}
    \textbf{Angular Velocity with Constant \\Acceleration:}
    \\ \( \vec{ \omega }_f = \vec{ \omega }_i + \vec{ \alpha } \Delta t \)
\end{minipage}
\\ \begin{minipage}{\textwidth}
    \textbf{Angular Displacement with Constant \\Acceleration:}
    \\ \( \theta_f = \theta_i + \vec{ \omega }_i \Delta t + \displaystyle\frac{ \vec{ \alpha } \Delta t^2 }{ 2 } \)
\end{minipage}
\\ \begin{minipage}{\textwidth}
    \textbf{Angular Velocity-Displacement \\Relation with Constant Acceleration:}
    \\ \( \vec{ \omega }_{f}^2 = \vec{ \omega }_{i}^2 + 2 \vec{ \alpha } \Delta \theta \)
\end{minipage}
\\ \begin{minipage}{\textwidth}
    \textbf{Angular to Linear Motion:}
    \\ \( \Delta x = r \Delta \theta \)
    \\ \( \vec{ v } = r \vec{ \omega } \)
    \\ \( \vec{ a }_T = r \vec{ \alpha } \)
\end{minipage}
\\ \begin{minipage}{\textwidth}
    \textbf{Torque:}
    \\ \( \vec{ \tau } = r \vec{ F } \sin \theta \)
\end{minipage}
\\ \begin{minipage}{\textwidth}
    \textbf{Archimedes's Law of Levers:}
    \\ \( \displaystyle\frac{ \vec{ F }_2 }{ \vec{ F }_1 } = \displaystyle\frac{ D_1 }{ D_2 } \)
\end{minipage}
\\ \begin{minipage}{\textwidth}
    \textbf{Moment of Inertia:}
    \\ \( I = Cmr^2 \)
    \\ \( I_{\text{sys}} = \displaystyle\sum{ C_i m_i r_i^2 } \)
\end{minipage}
\\ \begin{minipage}{\textwidth}
    \textbf{Parallel Axis Theorem:}
    \\ \( I' = I_{\text{cm}} + mx^2 \)
\end{minipage}
\\ \begin{minipage}{\textwidth}
    \textbf{Newton's Second Law for Rotational \\Motion:}
    \\ \( \vec{ \alpha } = \displaystyle\frac{ \vec{ \tau }_{\text{net}} }{ I_{\text{sys}} } \)
\end{minipage}
\\ \begin{minipage}{\textwidth}
    \textbf{Newton's Third Law for Rotational \\Motion:}
    \\ \( \Delta \vec{ L }_{a \rightarrow{b}} = - \Delta \vec{ L }_{b \rightarrow{a}} \)
\end{minipage}
\\ \begin{minipage}{\textwidth}
    \textbf{Rotational Kinetic Energy:}
    \\ \( k_{\text{rot}} = \displaystyle\frac{ I \vec{ \omega }^2 }{ 2 } \)
\end{minipage}
\\ \begin{minipage}{\textwidth}
    \textbf{Rotational Work:}
    \\ \( w = \vec{ \tau } \Delta \theta \)
\end{minipage}
\\ \begin{minipage}{\textwidth}
    \textbf{Rotational Work-Energy Theorem:}
    \\ \( w = \Delta k_{\text{rot}} \)
\end{minipage}
\\ \begin{minipage}{\textwidth}
    \textbf{Angular Momentum:}
    \\ \( \vec{ L } = I \vec{ \omega } \)
\end{minipage}
\\ \begin{minipage}{\textwidth}
    \textbf{Orbital Angular Momentum:}
    \\ \( \vec{ L } = rm \vec{ v } \sin \theta \)
\end{minipage}
\\ \begin{minipage}{\textwidth}
    \textbf{Angular Impulse-Momentum Theorem:}
    \\ \( \Delta \vec{ L } = I \Delta \vec{ \omega } = \vec{ \tau } \Delta t \)
\end{minipage}
\\ \begin{minipage}{\textwidth}
    \textbf{Conservation of Angular Momentum:}
    \\ \( \vec{ L }_f - \vec{ L }_i = 0 \)
\end{minipage}