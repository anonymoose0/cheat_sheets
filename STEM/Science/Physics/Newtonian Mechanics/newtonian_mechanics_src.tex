% Copyright (c)  2025  anonymoose0.
%    Permission is granted to copy, distribute and/or modify this document
%    under the terms of the GNU Free Documentation License, Version 1.3
%    or any later version published by the Free Software Foundation;
%    with no Invariant Sections, no Front-Cover Texts, and no Back-Cover Texts.
%    A copy of the license is included in the section entitled "GNU
%    Free Documentation License".

\begin{minipage}{\textwidth}
\textbf{Newton’s Second Law:}
    \\ \( \vec{ F }_{\text{net}} = m_{\text{sys}} \vec{ a } = \displaystyle\frac{ d \vec{ p } }{ dt } \)
\end{minipage}
\\ \begin{minipage}{\textwidth}
    \textbf{Newton’s Third Law: }
    \\ \( \vec{ F }_{a \rightarrow{b}} = - \vec{ F }_{b \rightarrow{a}} \)
\end{minipage}
\\ \begin{minipage}{\textwidth}
    \textbf{Definitions of Displacement, Velocity, \\and Acceleration:}
    \\ \( \Delta x = x_f - x_i \)
    \\ \( \vec{ v } = \displaystyle\frac{ \Delta x }{ \Delta t } \)
    \\ \( \vec{ a } = \displaystyle\frac{ \Delta \vec{ v } }{ \Delta t } \)
\end{minipage}
\\ \begin{minipage}{\textwidth}
    \textbf{Displacement with Constant \\Acceleration:}
    \\ \( x_f = x_i + \vec{ v }_x \Delta t + \displaystyle\frac{ \vec{ a }_x ( \Delta t )^2 }{ 2 } \)
\end{minipage}
\\ \begin{minipage}{\textwidth}
    \textbf{Velocity with Constant Acceleration:}
    \\ \( \vec{ v }_{xf} = \vec{ v }_{xi} + \vec{ a }_x \Delta t \)
\end{minipage}
\\ \begin{minipage}{\textwidth}
    \textbf{Velocity-Displacement Relation with \\Constant Acceleration:}
    \\ \( \vec{ v }_{xf}^2 = \vec{ v }_{xi}^2 + 2 \vec{ a }_x \Delta x \)
\end{minipage}
\\ \begin{minipage}{\textwidth}
    \textbf{Vector Equations:}
    \\ \( \vec{ A }_x = \vec{ A } \cos \theta \)
    \\ \( \vec{ A }_y = \vec{ A } \sin \theta \)
    \\ \( \vec{ A } = \sqrt{ \vec{ A }_{x}^2 + \vec{ A }_{y}^2 } \)
    \\ \( \theta = \arctan \displaystyle\frac{ \vec{ A }_y }{ \vec{ A }_x } \)
\end{minipage}
\\ \begin{minipage}{\textwidth}
    \textbf{Center of Mass:}
    \\ \( x_{cm} = \displaystyle\frac{ \displaystyle\sum{ m_i x_i } }{ \displaystyle\sum{ m_i } } \)
\end{minipage}
\\ \begin{minipage}{\textwidth}
    \textbf{Definition of Weight:}
    \\ \( \vec{ F }_w = m( g + \vec{ a }_y ) = -\vec{ F }_n \)
\end{minipage}
\\ \begin{minipage}{\textwidth}
    \textbf{Maximum Static Friction:}
    \\ \( \vec{ F }_{\text{sf max}} = \mu \vec{ F }_n \)
\end{minipage}
\\ \begin{minipage}{\textwidth}
    \textbf{Kinetic Friction:}
    \\ \( \vec{ F }_{\text{kf}} = \mu_k \vec{ F }_n \)
\end{minipage}
\\ \begin{minipage}{\textwidth}
    \textbf{Hooke's Law:}
    \\ \( \vec{ F }_{\text{sp x}} = -k \Delta x \)
\end{minipage}
\\ \begin{minipage}{\textwidth}
    \textbf{Newton's Law of Gravitation:}
    \\ \( \vec{ F }_g = \displaystyle\frac{ G m_1 m_2 }{ r^2 } \)
\end{minipage}
\\ \begin{minipage}{\textwidth}
    \textbf{Kepler's Third Law:}
    \\ \( t^2 = \displaystyle\frac{ 4 \pi^2 R^3 }{ MG } \)
\end{minipage}
\\ \begin{minipage}{\textwidth}
    \textbf{Time to Orbit:}
    \\ \( t = \displaystyle\frac{ 2 \pi r }{ \vec{ v } } \)
\end{minipage}
\\ \begin{minipage}{\textwidth}
    \textbf{Minimum Velocity to Orbit:}
    \\ \( \vec{ v }_{\text{min}} = \sqrt{ gr } \)
\end{minipage}
\\ \begin{minipage}{\textwidth}
    \textbf{Circular Acceleration:}
    \\ \( \vec{ a }_c = \displaystyle\frac{ \vec{ v }^2 }{ r } \)
\end{minipage}
\\ \begin{minipage}{\textwidth}
    \textbf{Work:}
    \\ \( w = \vec{ F }d \cos \theta \)
\end{minipage}
\\ \begin{minipage}{\textwidth}
    \textbf{Translational Kinetic Energy:}
    \\ \( k = \displaystyle\frac{ m\vec{ v }^2 }{ 2 } \)
\end{minipage}
\\ \begin{minipage}{\textwidth}
    \textbf{Gravitational Potential Energy:}
    \\ \( U_g = mgy \)
\end{minipage}
\\ \begin{minipage}{\textwidth}
    \textbf{Elastic Potential Energy:}
    \\ \( U_s = \displaystyle\frac{ k \Delta x^2 }{ 2 } \)
\end{minipage}
\\ \begin{minipage}{\textwidth}
    \textbf{Work-Energy Theorem:}
    \\ \( w = \Delta k \)
\end{minipage}
\\ \begin{minipage}{\textwidth}
    \textbf{Definition of Power:}
    \\ \( P = \displaystyle\frac{ \Delta E }{ \Delta t } = \displaystyle\frac{ w }{ \Delta t } = \vec{ F } \vec{ v } \cos \theta \)
\end{minipage}
\\ \begin{minipage}{\textwidth}
    \textbf{Definition of Impulse:}
    \\ \( \vec{ J } = \vec{ F }_{\text{avg}} \Delta t \)
\end{minipage}
\\ \begin{minipage}{\textwidth}
    \textbf{Definition of Momentum:}
    \\ \( \vec{ p } = m\vec{ v } \)
\end{minipage}
\\ \begin{minipage}{\textwidth}
    \textbf{Conservation of Momentum:}
    \\ \( \vec{ p }_f - \vec{ p }_i = 0 \)
\end{minipage}
\\ \begin{minipage}{\textwidth}
    \textbf{Impulse-Momentum Theorem:}
    \\ \( \vec{ J } = \Delta \vec{ p } = m \Delta \vec{ v } = \vec{ F } \Delta t \)
\end{minipage}
\\ \begin{minipage}{\textwidth}
    \textbf{Orbital Velocity:}
    \\ \( \vec{ v } = \sqrt{ \displaystyle\frac{ Gm }{ r } } \)
\end{minipage}
\\ \begin{minipage}{\textwidth}
    \textbf{Orbital Gravitational Potential Energy:}
    \\ \( U_g = \displaystyle\frac{ -G m_1 m_2 }{ r } \)
\end{minipage}
\\ \begin{minipage}{\textwidth}
    \textbf{Escape Velocity:}
    \\ \( \vec{ v }_{\text{esc}} = \sqrt{ \displaystyle\frac{ 2GM }{ r } } \)
\end{minipage}
\\ \begin{minipage}{\textwidth}
    \textbf{Period of a Pendulum:}
    \\ \( t_p = 2 \pi \sqrt{ \displaystyle\frac{ l }{ g } } \)
\end{minipage}
\\ \begin{minipage}{\textwidth}
    \textbf{Period of a Spring:}
    \\ \( t_s = 2 \pi \sqrt{ \displaystyle\frac{ m }{ k } } \)
\end{minipage}