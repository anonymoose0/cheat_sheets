% Copyright (c)  2025  anonymoose0.
%    Permission is granted to copy, distribute and/or modify this document
%    under the terms of the GNU Free Documentation License, Version 1.3
%    or any later version published by the Free Software Foundation;
%    with no Invariant Sections, no Front-Cover Texts, and no Back-Cover Texts.
%    A copy of the license is included in the section entitled "GNU
%    Free Documentation License".

\begin{minipage}{\textwidth}
    \textbf{Definition of the Riemann Integral:}
    \\ \( \displaystyle\int_{a}^{b} { f( x ) dx } = \displaystyle\lim_{n \rightarrow{\infty}} \displaystyle\sum_{i = 1}^{n} { f( x_{i}^{*} ) \Delta x ; \Delta x = \displaystyle\frac{ b - a }{ n } } \)
\end{minipage}
\\ \begin{minipage}{\textwidth}
    \textbf{Mean Value Theorem for Integrals:}
    \\ \( \forall f \exists c \in [a, b] : \displaystyle\frac{ \displaystyle\int_{a}^{b} { f( x ) dx } }{ b - a } = f( c ) \)
\end{minipage}
\\ \begin{minipage}{\textwidth}
    \textbf{Trigonometric Integrals:}
    \\ \( \displaystyle\int { \sin x dx } = - \cos x + C \)
    \\ \( \displaystyle\int { \cos x dx } = \sin x + C \)
    \\ \( \displaystyle\int { \tan x dx } = \ln| \sec x | + C \)
    \\ \( \displaystyle\int { \csc x dx } = \ln| \csc x - \cot x | + C \)
    \\ \( \displaystyle\int { \sec x dx } = \ln| \sec x + \tan x | + C \)
    \\ \( \displaystyle\int { \cot x dx } = - \ln| \csc x | + C \)
\end{minipage}
\\ \begin{minipage}{\textwidth}
    \textbf{Hyperbolic Trigonometric Integrals:}
    \\ \( \displaystyle\int { \sinh x dx } = \cosh x + C \)
    \\ \( \displaystyle\int { \cosh x dx } = \sinh x + C \)
    \\ \( \displaystyle\int { \tanh x dx } = \ln \cosh x + C \)
    \\ \( \displaystyle\int { \text{csch } x dx } = \ln \left| \tanh \displaystyle\frac{ x }{ 2 } \right| + C \)
    \\ \( \displaystyle\int { \text{sech } x dx } = \arctan \sinh x + C \)
    \\ \( \displaystyle\int { \coth x dx } = \ln| \sinh x | + C \)
\end{minipage}
\\ \begin{minipage}{\textwidth}
    \textbf{Inverse Trigonometric Integrals: }
    \\ \( \displaystyle\int { \arcsin x dx } = x \arcsin x + \sqrt{ 1 - x^2 } + C ; | x | \le 1 \)
    \\ \( \displaystyle\int { \arccos x dx } = x \arccos x - \sqrt{ 1 - x^2 } + C ; | x | \le 1 \)
    \\ \( \displaystyle\int { \arctan x dx } = x \arctan x - \displaystyle\frac{ \ln| 1 + x^2 | }{ 2 } + C \)
    \\ \( \displaystyle\int { \text{arccsc } x dx } = x \text{ arccsc } x + \ln \left| x \left( 1 + \sqrt{ 1 - x^{-2} } \right) \right| + C ; | x | \ge 1 \)
    \\ \( \displaystyle\int { \text{arcsec } x dx } = x \text{ arcsec } x - \ln \left| x \left( 1 + \sqrt{ 1 - x^{-2} } \right) \right| + C ; | x | \ge 1 \)
    \\ \( \displaystyle\int { \text{arccot } x dx } = x \text{ arccot } x + \displaystyle\frac{ \ln| 1 + x^2 | }{ 2 } + C \)
\end{minipage}