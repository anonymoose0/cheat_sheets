% Copyright (c)  2025  anonymoose0.
%    Permission is granted to copy, distribute and/or modify this document
%    under the terms of the GNU Free Documentation License, Version 1.3
%    or any later version published by the Free Software Foundation;
%    with no Invariant Sections, no Front-Cover Texts, and no Back-Cover Texts.
%    A copy of the license is included in the section entitled "GNU
%    Free Documentation License".

\begin{minipage}{\textwidth}
    \textbf{Fundamental Theorem of Calculus, Part One:}
    \\ \( f( x ) = \displaystyle\int_{0}^{h( x )} { g( t ) dt } \)
    \\ \( \implies \displaystyle\frac{ d }{ dx } f( x ) = ( g \circ h )( x ) \cdot \displaystyle\frac{ d }{ dx } h( x ) \)
\end{minipage}
\\ \begin{minipage}{\textwidth}
    \textbf{Fundamental Theorem of Calculus, Part Two:}
    \\ \( \displaystyle\int_{a}^{b} { \displaystyle\frac{ d }{ dx } f( x ) dx } = f( b ) - f( a ) \)
\end{minipage}

\subsection*{Limit}
\subimport{./Limit/}{limit_calculus_cheat_sheet_src.tex}

\subsection*{Differential}
\subimport{./Differential/}{differential_calculus_cheat_sheet_src.tex}

\subsection*{Integral}
\subimport{./Integral/}{integral_calculus_cheat_sheet.tex}

\subsection*{Implementations}
\begin{minipage}{\textwidth}
    \textbf{Arc Length:}
    \\ \( \left. L_{f( x )} \right|_{a}^{b} = \displaystyle\int_{a}^{b} { \sqrt{ 1 + \left( \displaystyle\frac{ d }{ dx } f( x ) \right)^2 } dx } \)
\end{minipage}
\\ \begin{minipage}{\textwidth}
    \textbf{Surface Area of a Rotational Solid:}
    \\ \( \left. S_{f( x )} \right|_{a}^{b} = 2 \pi \displaystyle\int_{a}^{b} { r \cdot \sqrt{ 1 + \left( \displaystyle\frac{ d }{ dx } f( x ) \right)^2 } dx } \)
\end{minipage}
\\ \begin{minipage}{\textwidth}
    \textbf{Volume of a Rotational Solid:}
    \\ \( \left. V_{f( x )} \right|_{a}^{b} = 2 \pi \displaystyle\int_{a}^{b} { f( x )^2 dx } \)
\end{minipage}
\\ \begin{minipage}{\textwidth}
    \textbf{Definition of Euler's Number:}
    \\ \( e = \displaystyle\lim_{n \rightarrow{\infty}} { \left( 1 + \displaystyle\frac{ 1 }{ n } \right)^n } \)
\end{minipage}
\\ \begin{minipage}{\textwidth}
    \textbf{Exponential McLauren Series:}
    \\ \( e^x = \displaystyle\sum_{n = 0}^{\infty} { \displaystyle\frac{ x^n }{ n! } } \)
\end{minipage}
\\ \begin{minipage}{\textwidth}
    \textbf{Euler's Formula:}
    \\ \( e^{i\theta} = \cos{\theta} + i\sin{\theta} \)
\end{minipage}
\\ \begin{minipage}{\textwidth}
    \textbf{Euler's Identity:}
    \\ \( e^{i \pi} + 1 = 0 \)
    \\ \( e^{i\tau} = 1  \)
\end{minipage}
\\ \begin{minipage}{\textwidth}
    \textbf{Power Series of Trigonometric Functions: }
    \\ \( \sin x = \displaystyle\sum_{n = 0}^{\infty} { \displaystyle\frac{ ( -1 )^n }{ ( 2n + 1 )! } \cdot x^{2n + 1} } \)
    \\ \( \cos x = \displaystyle\sum_{n = 0}^{\infty} { \displaystyle\frac{ ( -1 )^n }{ (2n)! } \cdot x^{2n} } \)
    \\ \( \tan x = \displaystyle\sum_{n = 0}^{\infty} { \displaystyle\frac{ U_{2n + 1} }{ ( 2n + 1 )! } \cdot x^{2n + 1} } ;  | x | < \displaystyle\frac{ \tau }{ 4 } \)
    \\ \( \csc x = \displaystyle\sum_{n = 0}^{\infty} { \displaystyle\frac{ ( -1 )^n 2( 2^{2n - 1} - 1 ) B_{2n} }{ ( 2n + 1 )! } \cdot x^{2n} } ; 0 < x < \displaystyle\frac{ \tau }{ 2 } \)
    \\ \( \sec x = \displaystyle\sum_{n = 0}^{\infty} { \displaystyle\frac{ U_{2n} }{ ( 2n )! } \cdot x^{2n} } ; | x | < \displaystyle\frac{ \tau }{ 4 } \)
    \\ \( \cot x = \displaystyle\sum_{n = 0}^{\infty} { \displaystyle\frac{ ( -1 )^n 2^{2n - 1} B_{2n} }{ ( 2n + 1 )! } \cdot x^{2n + 1} } ; 0 < x < \displaystyle\frac{ \tau }{ 2 } \)
\end{minipage}