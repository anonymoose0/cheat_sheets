% Copyright (c)  2025  anonymoose0.
%    Permission is granted to copy, distribute and/or modify this document
%    under the terms of the GNU Free Documentation License, Version 1.3
%    or any later version published by the Free Software Foundation;
%    with no Invariant Sections, no Front-Cover Texts, and no Back-Cover Texts.
%    A copy of the license is included in the section entitled "GNU
%    Free Documentation License".

\begin{minipage}{\textwidth}
    \textbf{Definition of Sine:}
    \\ \( \sin A = \displaystyle\frac{ a }{ c } \)
\end{minipage}
\\ \begin{minipage}{\textwidth}
    \textbf{Definition of Cosine:}
    \\ \( \cos A = \displaystyle\frac{ b }{ c } \)
\end{minipage}
\\ \begin{minipage}{\textwidth}
    \textbf{Definition of Tangent:}
    \\ \( \tan A = \displaystyle\frac{ a }{ b } \)
\end{minipage}
\\ \begin{minipage}{\textwidth}
    \textbf{Definition of Cosecant:}
    \\ \( \csc A = \displaystyle\frac{ c }{ a } \)
\end{minipage}
\\ \begin{minipage}{\textwidth}
    \textbf{Definition of Secant:}
    \\ \( \sec A = \displaystyle\frac{ c }{ b } \)
\end{minipage}
\\ \begin{minipage}{\textwidth}
    \textbf{Definition of Cotangent:}
    \\ \( \cot A = \displaystyle\frac{ b }{ a } \)
\end{minipage}
\\ \begin{minipage}{\textwidth}
    \textbf{Law of Sines:}
    \\ \( \displaystyle\frac{ \sin A }{ a } = \displaystyle\frac{ \sin B }{ b } = \displaystyle\frac{ \sin C }{ c } \)
\end{minipage}
\\ \begin{minipage}{\textwidth}
    \textbf{Law of Cosines:}
    \\ \( c^2 = a^2 + b^2 - 2ab \cos C \)
\end{minipage}
\\ \begin{minipage}{\textwidth}
    \textbf{Law of Tangents:}
    \\ \( \displaystyle\frac{ \tan \frac{ A - B }{ 2 } }{ \tan \frac{ A + B }{ 2 } } = \displaystyle\frac{ a - b }{ a + b } \)
\end{minipage}
\\ \begin{minipage}{\textwidth}
    \textbf{Law of Cotangents:}
    \\ \( s = \displaystyle\frac{ a + b + c }{ 2 } \)
    \\ \( \displaystyle\frac{ \cot \frac{ A }{ 2 } }{ s - a } = \displaystyle\frac{ \cot \frac{ B }{ 2 } }{ s - b } = \displaystyle\frac{ \cot \frac{ C }{ 2 } }{ s - c } \)
\end{minipage}
\\ \begin{minipage}{\textwidth}
    \textbf{Pythagorean Identities:}
    \\ \( \sin^2 x + \cos^2 x = 1 \)
    \\ \( \tan^2 x + 1 = \sec^2 x \)
    \\ \( 1 + \cot^2 x = \csc^2 x \)
\end{minipage}
\\ \begin{minipage}{\textwidth}
\textbf{Sum/Difference Identities:}
    \\ \( \sin( x \pm y ) = \sin( x ) \cdot \cos( y ) \pm \cos( x ) \cdot \sin( y ) \)
    \\ \( \cos( x \pm y ) = \cos( x ) \cdot \cos( y ) \mp \sin( x ) \cdot \sin( y ) \)
    \\ \( \tan( x \pm y ) = \displaystyle\frac{ \tan( x ) \pm \tan( y ) }{ 1 \mp \tan( x ) \cdot \tan( y ) } \)
    \\ \( \csc( x \pm y ) = \displaystyle\frac{ \sec( x ) \sec( y ) \csc( x ) \csc( y ) }{ \sec( x ) \csc( y ) \pm \csc( x ) \sec( y ) } \)
    \\ \( \sec( x \pm y ) = \displaystyle\frac{ \sec( x ) \sec( y ) \csc( x ) \csc( y ) }{ \csc( x ) \csc( y ) \mp \sec( x ) \sec( y ) } \)
    \\ \( \cot( x \pm y ) = \displaystyle\frac{ \cot( x ) \cot( y ) \mp 1 }{ \cot( y ) \pm \cot( x ) } \)    
\end{minipage}
\\ \begin{minipage}{\textwidth}
    \textbf{Double Angle Identities:}
    \\ \( \sin( 2x ) = 2 \sin( x ) \cdot \cos( x ) \)
    \\ \( \cos( 2x ) = \cos^2( x ) - \sin^2( x ) \)
    \\ \( \tan( 2x ) = \displaystyle\frac{ 2 \tan( x ) }{ 1 - \tan^2( x ) } \)
    \\ \( \csc( 2x ) = \displaystyle\frac{ \sec( x ) \cdot \csc( x ) }{ 2 } \)
    \\ \( \sec( 2x ) = \displaystyle\frac{ \sec^2( x ) }{ 2 - \sec^2( x ) } \)
    \\ \( \cot( 2x ) = \displaystyle\frac{ \cot^2( x ) - 1 }{ 2 \cot( x ) } \)
\end{minipage}
\\ \begin{minipage}{\textwidth}
    \textbf{Half Angle Identities:}
    \\ \( \sin \left( \displaystyle\frac{ x }{ 2 } \right) = \text{sgn} \left( \sin \left( \displaystyle\frac{ x }{ 2 } \right) \right) \sqrt{ \displaystyle\frac{ 1 - \cos( x ) }{ 2 } } \)
    \\ \( \cos \left( \displaystyle\frac{ x }{ 2 } \right) = \text{sgn} \left( \cos \left( \displaystyle\frac{ x }{ 2 } \right) \right) \sqrt{ \displaystyle\frac{ 1 + \cos( x ) }{ 2 } } \)
    \\ \( \tan \left( \displaystyle\frac{ x }{ 2 } \right) = \displaystyle\frac{ \sin( x ) }{ 1 + \cos( x ) } \)
    \\ \( \csc \left( \displaystyle\frac{ x }{ 2 } \right) = \text{sgn} \left( \sin \left( \displaystyle\frac{ x }{ 2 } \right) \right) \sqrt{ \displaystyle\frac{ 2 }{ 1 - \cos( x ) } } \)
    \\ \( \sec \left( \displaystyle\frac{ x }{ 2 } \right) = \text{sgn} \left( \cos \left( \displaystyle\frac{ x }{ 2 } \right) \right) \sqrt{ \displaystyle\frac{ 2 }{ 1 + \cos( x ) } } \)
    \\ \( \cot \left( \displaystyle\frac{ x }{ 2 } \right) = \displaystyle\frac{ 1 + \cos( x ) }{ \sin( x ) } \)
\end{minipage}
\\ \begin{minipage}{\textwidth}
    \textbf{Product to Sum Identities:}
    \\ \( \sin( x ) \cdot \sin( y ) = \displaystyle\frac{ \cos( x - y ) - \cos( x + y ) }{ 2 } \)
    \\ \( \cos( x ) \cdot \cos( y ) = \displaystyle\frac{ \cos( x - y ) + \cos( x + y ) }{ 2 } \)
    \\ \( \sin( x ) \cdot \cos( y ) = \displaystyle\frac{ \sin( x - y ) - \sin( x + y ) }{ 2 } \)
    \\ \( \tan( x ) \cdot \tan( y ) = \displaystyle\frac{ \cos( x - y ) - \cos( x + y ) }{ \cos( x - y ) + \cos( x + y ) } \)
    \\ \( \tan( x ) \cdot \cot( y ) = \frac{ \cos( x - y ) + \cos( x + y ) }{ \cos( x - y ) - \cos( x + y ) } \)
\end{minipage}
\\ \begin{minipage}{\textwidth}
    \textbf{Sum to Product Identities:}
    \\ \( \sin( x ) \pm \sin( y ) = 2 \sin \left( \displaystyle\frac{ x \pm y }{ 2 } \right) \cos \left( \displaystyle\frac{ x \mp y }{ 2 } \right) \)
    \\ \( \cos( x ) + \cos( y ) = 2 \cos \left( \displaystyle\frac{ x + y }{ 2 } \right) \cos \left( \displaystyle\frac{ x - y }{ 2 } \right) \)
    \\ \( \cos( x ) - \cos( y ) = -2 \sin \left( \displaystyle\frac{ x + y }{ 2 } \right) \sin \left( \displaystyle\frac{ x - y }{ 2 } \right) \)
    \\ \( \tan( x ) \pm \tan( y ) = \frac{ \sin( x \pm y ) }{ \cos( x ) \cdot \cos( y ) } \)
\end{minipage}
\\ \begin{minipage}{\textwidth}
    \textbf{Polar Coordinate Equations:}
    \\ \( x^2 + y^2 = r^2 \)
    \\ \( \tan \theta = \displaystyle\frac{ y }{ x } \)
    \\ \( x = r \cos \theta \)
    \\ \( y = r \sin \theta \)
\end{minipage}
\\ \begin{minipage}{\textwidth}
    \textbf{Polar Form of Conic Sections:}
    \\ \( r = \displaystyle\frac{ de }{ 1 + e \sin \theta } \)
    \\ \( r = \displaystyle\frac{ de }{ 1 + e \cos \theta } \)
\end{minipage}