% Copyright (c)  2025  anonymoose0.
%    Permission is granted to copy, distribute and/or modify this document
%    under the terms of the GNU Free Documentation License, Version 1.3
%    or any later version published by the Free Software Foundation;
%    with no Invariant Sections, no Front-Cover Texts, and no Back-Cover Texts.
%    A copy of the license is included in the section entitled "GNU
%    Free Documentation License".

\documentclass[12pt]{article}
\usepackage[utf8]{inputenc}
\usepackage{amsmath, amssymb, amsthm}
\usepackage{fullpage}

\begin{document}
\begin{center}
    
\section*{\textbf{\underline{Geometry}}}
\textbf{}
\begin{tabular}{ | p{\textwidth} |}
    \hline

    \hline
\end{tabular}


\section*{\textbf{\underline{Algebra}}}
\textbf{The Fundamental Theorem of Algebra}
\begin{tabular}{ | p{\textwidth} |}
    \hline
    If a function \( f(x) \) is a polynomial, i.e. \( f(x) = \displaystyle\sum_{k=0}^{n} {c_k x^k} \) where \( c \in \mathbb{C} \) then \( \exists r_k \in \mathbb{C}\) such that \( f(r_k) = 0 \) \\
    \hline
\end{tabular}


\section*{\textbf{\underline{Calculus}}}
\textbf{The Squeeze Theorem}
\begin{tabular}{ | p{\textwidth} |}
    \hline

    \hline
\end{tabular}
\textbf{The Differential Mean Value Theorem}
\begin{tabular}{ | p{\textwidth} |}
    \hline

    \hline
\end{tabular}
\textbf{The Integral Mean Value Theorem}
\begin{tabular}{ | p{\textwidth} |}
    \hline

    \hline
\end{tabular}


\end{center}
\end{document}