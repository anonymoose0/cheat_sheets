% Copyright (c)  2025  anonymoose0.
%    Permission is granted to copy, distribute and/or modify this document
%    under the terms of the GNU Free Documentation License, Version 1.3
%    or any later version published by the Free Software Foundation;
%    with no Invariant Sections, no Front-Cover Texts, and no Back-Cover Texts.
%    A copy of the license is included in the section entitled "GNU
%    Free Documentation License".

\documentclass[12pt]{article}
\usepackage[utf8]{inputenc}
\usepackage{amsmath, amssymb, amsthm}
\usepackage{fullpage}

\begin{document}
\begin{center}
    
\section*{\textbf{\underline{Geometry}}}
\subsection*{\textbf{}}
\begin{tabular}{ | p{\textwidth} |}
    \hline

    \hline
\end{tabular}


\section*{\textbf{\underline{Algebra}}}
\subsection*{\textbf{The Fundamental Theorem of Algebra}}
\begin{tabular}{ | p{\textwidth} |}
    \hline
    If a function \( f(x) \) is a polynomial, i.e. \( f(x) = \displaystyle\sum_{k=0}^{n} {c_k x^k} \) where \( c \in \mathbb{C} \) then there exists \( r_k \in \mathbb{C}\)
    such that \( f(r_k) = 0 \) \\
    \hline
\end{tabular}


\section*{\textbf{\underline{Calculus}}}
\subsection*{\textbf{The Squeeze Theorem}}
\begin{tabular}{ | p{\textwidth} |}
    \hline
    For any function \( f(x) \) continuous on \( [a, b] \) where the also exist functions \( g(x) \) and \( h(x) \) such that \( h(x) \le f(x) \le g(x) \forall x \in [a, b] \),
    if \( f(x) \) is discontinuous at point \( c \in [a, b] \) and \( \displaystyle\lim_{x \rightarrow{c}} {g(x)} = \displaystyle\lim_{x \rightarrow{c}} {h(x)} = L \),
    then \( \displaystyle\lim_{x \rightarrow{c}} {f(x)} = L \) \\
    \hline
\end{tabular} 

\subsection*{\textbf{The Differential Mean Value Theorem}}
\begin{tabular}{ | p{\textwidth} |}
    \hline
    For any continuous function \( f(x) \) that is defined on \( [a, b] \) and differentiable on \( (a, b) \) where \( b > a \) then there exists \( c \in (a, b) \)
    such that \( \displaystyle\frac{ d }{ dx } f(c) = \displaystyle\frac{ f(b) - f(a) }{ b - a } \) \\
    \hline
\end{tabular} 

\subsection*{\textbf{The Integral Mean Value Theorem}}
\begin{tabular}{ | p{\textwidth} |}
    \hline
    For any continuous function \( f(x) \) that is defined on \( [a, b] \) where \( b > a \) then there exists \( c \in (a, b) \)
    such that \( f(c) = \displaystyle\frac{\displaystyle\int_{a}^{b} { f(x) dx } }{ b - a } \) \\
    \hline
\end{tabular} 


\end{center}
\end{document}