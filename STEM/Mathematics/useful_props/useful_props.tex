% Copyright (c)  2025  anonymoose0.
%    Permission is granted to copy, distribute and/or modify this document
%    under the terms of the GNU Free Documentation License, Version 1.3
%    or any later version published by the Free Software Foundation;
%    with no Invariant Sections, no Front-Cover Texts, and no Back-Cover Texts.
%    A copy of the license is included in the section entitled "GNU
%    Free Documentation License".

\documentclass[12pt]{article}
\usepackage[utf8]{inputenc}
\usepackage{amsmath, amssymb, amsthm}
\usepackage{fullpage}

\begin{document}
\twocolumn

\section*{\underline{\textbf{Equality}}}
\begin{minipage}{\textwidth}
    \end{minipage}
    \begin{minipage}{\textwidth}
    xtbf{Reflexive Property:}
    \\ \( (a = b) \iff (b = a) \)
\end{minipage}
\begin{minipage}{\textwidth}
    \textbf{Transitive Property:}
    \\ \( (a = b) \land (b = c) \implies (a = c) \)
\end{minipage}



\section*{\underline{\textbf{Addition}}}
\begin{minipage}{\textwidth}
    \textbf{Commutative Property:}
    \\ \( a + b = b + a \)
\end{minipage}
\begin{minipage}{\textwidth}
    \textbf{Associative Property:}
    \\ \( (a + b) + c = a + (b + c) \)
\end{minipage}
\begin{minipage}{\textwidth}
    \textbf{Additive Identity Property:}
    \\ \( a + 0 = a \)
\end{minipage}
\begin{minipage}{\textwidth}
    \textbf{Inverse Property:}
    \\ \( a + (-a) = 0 \)
\end{minipage}



\section*{\underline{\textbf{Multiplication}}}
\begin{minipage}{\textwidth}
    \textbf{Commutative Property:}
    \\ \( a \cdot b = b \cdot a \)
\end{minipage}
\begin{minipage}{\textwidth}
    \textbf{Associative Property:}
    \\ \( (a \cdot b) \cdot = a \cdot (b \cdot c) \)
\end{minipage}
\begin{minipage}{\textwidth}
    \textbf{Multiplicative Identity Property:}
    \\ \( a \cdot 1 = a \)
\end{minipage}
\begin{minipage}{\textwidth}
    \textbf{Inverse Property:}
    \\ \( a \cdot \displaystyle\frac{ 1 }{ a } = 1 ; a \ne 0 \)
\end{minipage}
\begin{minipage}{\textwidth}
    \textbf{Distributive Property:}
    \\ \( a \cdot (b + c) = a \cdot b + a \cdot c \)
\end{minipage}
\begin{minipage}{\textwidth}
    \textbf{Zero Product Property:}
    \\ \( a \cdot 0 = 0 \)
\end{minipage}



\section*{\underline{\textbf{Exponents}}}
\begin{minipage}{\textwidth}
    \textbf{Exponential Identity Property:}
    \\ \( a^1 = a \)
\end{minipage}
\begin{minipage}{\textwidth}
    \textbf{Zero Power Property:}
    \\ \( a^0 = 1 ; a \ne 0 \)
\end{minipage}
\begin{minipage}{\textwidth}
    \textbf{Power of One Property:}
    \\ \( 1^a = a \)
\end{minipage}
\begin{minipage}{\textwidth}
    \textbf{Power of Zero Property:}
    \\ \( 0^a = 0 ; a \ne 0 \)
\end{minipage}
\begin{minipage}{\textwidth}
    \textbf{Product of Powers Property:}
    \\ \( x^a \cdot x^b = x^{a + b} \)
\end{minipage}
\begin{minipage}{\textwidth}
    \textbf{Quotient of Powers Property:}
    \\ \( \displaystyle\frac{ x^a }{ x^b } = x^{a - b} ; x \ne 0 \)
\end{minipage}
\begin{minipage}{\textwidth}
    \textbf{Power of a Power Property:}
    \\ \( (x^a)^b = x^{a \cdot b} \)
\end{minipage}
\begin{minipage}{\textwidth}
    \textbf{Power of a Product Property:}
    \\ \( (x \cdot y)^a = x^a \cdot y^a \)
\end{minipage}
\begin{minipage}{\textwidth}
    \textbf{Power of a Quotient Property:}
    \\ \( \left( \displaystyle\frac{ x }{ y } \right)^a = \frac{ x^a }{ y^a } ; y \ne 0 \)
\end{minipage}
\begin{minipage}{\textwidth}
    \textbf{Negative Power Property:}
    \\ \( x^{-a} = \frac{ 1 }{ x^a } ; x \ne 0 \)
\end{minipage}
\begin{minipage}{\textwidth}
    \textbf{Fractional Power Property:}
    \\ \( x^{\frac{ a }{ b }} = \displaystyle\sqrt[b]{x^a} ; b \ne 0 \)
\end{minipage}



\section*{\underline{\textbf{Roots}}}
\begin{minipage}{\textwidth}
    \textbf{Product of Roots Property:}
    \\ \( \displaystyle\sqrt[a]{x} \cdot \displaystyle\sqrt[a]{y} = \displaystyle\sqrt[a]{x \cdot y} ; a \ne 0 \land x, y \in \mathbb{R}^+ \)
\end{minipage}
\begin{minipage}{\textwidth}
    \textbf{Quotient of Roots:}
    \\ \( \displaystyle\frac{ \displaystyle\sqrt[a]{x} }{ \displaystyle\sqrt[a]{y} } = \displaystyle\sqrt[a]{\displaystyle\frac{ x }{ y }} ; a, y \ne 0 \)
\end{minipage}
\begin{minipage}{\textwidth}
    \textbf{Radical Identity Property:}
    \\ \( \displaystyle\sqrt[a]{x^a} = x ; a \ne 0 \)
\end{minipage}



\section*{\underline{\textbf{Logarithms}}}
\begin{minipage}{\textwidth}
    \textbf{Logarithmic Identity Properties:}
    \\ \( \log_x{1} = 0 \)
    \\ \( \log_x{x} = 1 \)
\end{minipage}
\begin{minipage}{\textwidth}
    \textbf{Product Property:}
    \\ \( log_a{(x \cdot y)} = \log_a{x} + \log_a{y} \)
\end{minipage}
\begin{minipage}{\textwidth}
    \textbf{Quotient Property:}
    \\ \( \log_a{\displaystyle\frac{ x }{ y }} = \log_a{x} - \log_a{y} \)
\end{minipage}
\begin{minipage}{\textwidth}
    \textbf{Power Property:}
    \\ \( \log_a{x^y} = y \cdot \log_a{x} \)
\end{minipage}
\begin{minipage}{\textwidth}
    \textbf{Base Change Property:}
    \\ \( \log_a{x} = \displaystyle\frac{ \log_n{x} }{ \log_n{a} } \)
\end{minipage}



\section*{\underline{\textbf{Summation}}}
\begin{minipage}{\textwidth}
    \textbf{Constant Factorization:}
    \\ \( \displaystyle\sum_{k=1}^{n} {c \cdot a_k } = c \cdot \displaystyle\sum_{k=1}^{n} { a_k } \)
\end{minipage}
\begin{minipage}{\textwidth}
    \textbf{Sum or Difference of Sequences:}
    \\ \( \displaystyle\sum_{k = 1}^{n} {(a_k \pm b_k)} = \displaystyle\sum_{k = 1}^{n} {a_k} \pm \displaystyle\sum_{k = 1}^{n} {b_k} \)
\end{minipage}
\begin{minipage}{\textwidth}
    \textbf{Summation of a Constant:}
    \\ \( \displaystyle\sum_{k = 1}^{n} {c} = n \cdot c \)
\end{minipage}
\begin{minipage}{\textwidth}
    \textbf{Index Shift:}
    \\ \( \displaystyle\sum_{k = 1}^{n} {a_k} = \displaystyle\sum_{k = 1 + p}^{n + p} {a_{k-p}} \)
\end{minipage}



\section*{\underline{\textbf{Products}}}
\begin{minipage}{\textwidth}
    \textbf{Associative Property:}
    \\ \( \displaystyle\prod_{k = 1}^{n} {(a_k \cdot b_k)} = \left( \displaystyle\prod_{k = 1}^{n} {a_k} \right) \cdot \left( \displaystyle\prod_{k = 1}^{n} {b_k} \right) \)
\end{minipage}
\begin{minipage}{\textwidth}
    \textbf{Commutative Property:}
    \\ \( \left( \displaystyle\prod_{k = 1}^{n} {a_k} \right)^x = \displaystyle\prod_{k = 1}^{n} {a_k^x} \)
\end{minipage}



\section*{\underline{\textbf{Limits}}}
\begin{minipage}{\textwidth}
    \textbf{Sum of Limits}
    \\ \( \displaystyle\lim_{x \rightarrow{c}} {(f + g)(x)} = \displaystyle\lim_{x \rightarrow{c}} {f(x)} + \displaystyle\lim_{x \rightarrow{c}} {g(x)} \)
\end{minipage}
\begin{minipage}{\textwidth}
    \textbf{Difference of Limits:}
    \\ \( \displaystyle\lim_{x \rightarrow{c}} {(f - g)(x)} = \displaystyle\lim_{x \rightarrow{c}} {f(x)} - \displaystyle\lim_{x \rightarrow{c}} {g(x)} \)
\end{minipage}
\begin{minipage}{\textwidth}
    \textbf{Product of Limits:}   
    \\ \( \displaystyle\lim_{x \rightarrow{c}} {(f \cdot g)(x)} = \displaystyle\lim_{x \rightarrow{c}} {f(x)} \cdot \displaystyle\lim_{x \rightarrow{c}} {g(x)} \)
\end{minipage}
\begin{minipage}{\textwidth}
    \textbf{Quotient of Limits:}
    \\ \( \displaystyle\lim_{x \rightarrow{c}} {\displaystyle\frac{ f(x) }{ g(x) }} = \displaystyle\frac{ \displaystyle\lim_{x \rightarrow{c}} {f(x)} }{ \displaystyle\lim_{x \rightarrow{c}} {g(x)} } ; \displaystyle\lim_{x \rightarrow{c}} {g(x)} \ne 0 \)
\end{minipage}
\begin{minipage}{\textwidth}
    \textbf{Power of Limits:}
    \\ \( \displaystyle\lim_{x \rightarrow{c}} {f(x)^{\frac{ n }{ d }}} = (\displaystyle\lim_{x \rightarrow{c}} {f(x)})^{\frac{ n }{ d }} ; \displaystyle\frac{ n }{ d } \in \mathbb{R} \)
\end{minipage}
\begin{minipage}{\textwidth}
    \textbf{Composition of Limits:}
    \\ \( \displaystyle\lim_{x \rightarrow{c}} {(f \circ g)(x)} = f(\displaystyle\lim_{x \rightarrow{c}} {g(x)}) \)
\end{minipage}
\begin{minipage}{\textwidth}
    \textbf{Constant Multiple Rule:}
    \\ \( \displaystyle\lim_{x \rightarrow{c}} {(k \cdot f(x))} = k \cdot \displaystyle\lim_{x \rightarrow{c}} {f(x)} \)
\end{minipage}
\begin{minipage}{\textwidth}
    \textbf{Constant Rule:}
    \\ \( \displaystyle\lim_{x \rightarrow{c}} {k} = k \)
\end{minipage}
\begin{minipage}{\textwidth}
    \textbf{Identity Rule:}
    \\ \( \displaystyle\lim_{x \rightarrow{c}} {x} = c \)
\end{minipage}



\section*{\underline{\textbf{Derivatives}}}
\begin{minipage}{\textwidth}
    \textbf{Power Rule:}
    \\ \( \displaystyle\frac{ d }{ dx } x^n = n \cdot x^{n - 1} \)
\end{minipage}
\begin{minipage}{\textwidth}
    \textbf{Functional Power Rule:}
    \\ \( \displaystyle\frac{ d }{ dx } \left( f(x)^{g(x)} \cdot \right) = f(x)^{g(x)} \left( \displaystyle\frac{ g(x) \cdot \frac{ df }{ dx } }{ f(x) } + \displaystyle\frac{ dg }{ dx } \cdot \ln{f(x)} \right) \)
\end{minipage}
\begin{minipage}{\textwidth}
    \textbf{Sum Rule:}
    \\ \( \displaystyle\frac{ d }{ dx } (f + g)(x) = \displaystyle\frac{ df }{ dx } + \displaystyle\frac{ dg }{ dx } \)
\end{minipage}
\begin{minipage}{\textwidth}
    \textbf{Product Rule:}
    \\ \( \displaystyle\frac{ d }{ dx } (f \cdot g)(x) = f(x) \cdot \displaystyle\frac{ dg }{ dx } + g(x) \cdot \displaystyle\frac{ df }{ dx } \)
\end{minipage}
\begin{minipage}{\textwidth}
    \textbf{General Leibniz Rule:}
    \\ \( \displaystyle\frac{ d^n }{ dx^n } (f \cdot g)(x) = \displaystyle\sum_{k = 0}^{n} {\binom{n}{k} \displaystyle\frac{ d^{n - k} f }{ dx^{n - k} }} \cdot \displaystyle\frac{ d^k g }{ dx^k } \)
\end{minipage}
\begin{minipage}{\textwidth}
    \textbf{Quotient Rule:}
    \\ \( \displaystyle\frac{ d }{ dx } \left( \displaystyle\frac{ f(x) }{ g(x) } \right) = \displaystyle\frac{ g(x) \cdot \frac{ df }{ dx } - f(x) \frac{ dg }{ dx } }{ g(x)^2 } \)
\end{minipage}
\begin{minipage}{\textwidth}
    \textbf{Chain Rule:}
    \\ \( \displaystyle\frac{ d }{ dx } (f \circ g)(x) = \displaystyle\frac{ df }{ dg } \cdot \displaystyle\frac{ dg }{ dx } \)
\end{minipage}
\begin{minipage}{\textwidth}
    \textbf{Exponential Derivatives:}
    \\ \( \displaystyle\frac{ d }{ dx } n^x = n^x \ln(n) ; n > 0 \)
\end{minipage}
\begin{minipage}{\textwidth}
    \textbf{Logarithmic Derivatives:}
    \\ \( \displaystyle\frac{ d }{ dx } \log_b{x} = \displaystyle\frac{ 1 }{ x \ln b } \)
\end{minipage}



\section*{\underline{\textbf{Definite Integrals}}}
\begin{minipage}{\textwidth}
    \textbf{Sum of Integrals:}
    \\ \( \displaystyle\int_{a}^{c} {f(x) dx} = \displaystyle\int_{a}^{b} {f(x) dx} + \displaystyle\int_{b}^{c} {f(x) dx} \)
\end{minipage}
\begin{minipage}{\textwidth}
    \textbf{Identity Rule:}
    \\ \( \displaystyle\int_{a}^{a} {f(x) dx} = 0 \)
\end{minipage}
\begin{minipage}{\textwidth}
    \textbf{Inverse Rule:}
    \\ \( \displaystyle\int_{a}^{b} {f(x) dx} = - \displaystyle\int_{b}^{a} {f(x) dx} \)
\end{minipage}



\section*{\underline{\textbf{Indefinite Integrals}}}
\begin{minipage}{\textwidth}
    \textbf{Inverse Power Rule:}
    \\ \( \displaystyle\int {x^n dx} = \displaystyle\frac{ x^{n + 1} }{ n + 1 } + C ; n \ne -1 \)
\end{minipage}
\begin{minipage}{\textwidth}
    \textbf{Sum Rule:}
    \\ \( \displaystyle\int {(f + g)(x) dx} = \displaystyle\int {f(x) dx} + \displaystyle\int {g(x) dx} \)
\end{minipage}
\begin{minipage}{\textwidth}
    \textbf{Integration by Parts:}
    \\ \( \displaystyle\int {(f \cdot g)(x) dx} = f(x) \cdot \displaystyle\int {g(x) dx} - \int {g(x) \cdot \frac{ df }{ dx } dx} \)
    \\ \( \displaystyle\int {u dv} = u \cdot v - \displaystyle\int {v du} \)
\end{minipage}
\begin{minipage}{\textwidth}
    \textbf{Integration by Substitution:}
    \\ \( \displaystyle\int {\displaystyle\frac{ dg }{ dx } \cdot (f \circ g)(x) dx} = \displaystyle\int {f(u) du} ; u = g(x) \)
\end{minipage}
\begin{minipage}{\textwidth}
    \textbf{Exponential Integrals:}
    \\ \( \displaystyle\int {n^x dx} = \displaystyle\frac{ n^x }{ \ln(n) } + C \)
\end{minipage}
\begin{minipage}{\textwidth}
    \textbf{Logarithmic Integrals:}
    \\ \( \displaystyle\int {\log_b{x} dx} = \displaystyle\frac{ x }{ \ln b } (\ln x - 1) + C \)
\end{minipage}



\end{document}